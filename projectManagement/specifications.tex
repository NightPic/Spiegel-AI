\documentclass{article}
\usepackage{enumitem}

\title{Lastenheft für das Projekt Spiegel AI}

\author{
  Gruppe 5 \\ \\
  Leon Kranner \\
  \texttt{Matrikelnummer: 3276114} \\ \\
  Marco Kuner \\
  \texttt{Matrikelnummer: 3290598} \\ \\
  David Vollmer \\
  \texttt{Matrikelnummer: 3314252} \\ \\
  Marcel Wagner \\
  \texttt{Matrikelnummer: 3328254}
}

\date{\today}


% Bei neuer Version des Lastenhefts bitte die Änderungshistorie anpassen. Danke :)

\begin{document}
\maketitle


\section{Einführung}
Dieses Lastenheft beschreibt die Anforderungen und Spezifikationen für die Entwicklung eines Smart-Spiegels. Der Smart-Spiegel ist ein innovatives Gerät, das traditionelle Spiegeltechnologie mit moderner Informationstechnologie kombiniert, um dem Benutzer eine Vielzahl von Funktionen und Diensten zu bieten. Entwickelt wird dieser Smart-Spiegel, um den täglichen Lebensstil der Benutzer zu bereichern, indem er relevante Informationen und Dienste direkt während des Morgenrituals im Badezimmer oder an anderen strategischen Standorten im Haushalt zur Verfügung stellt. Dieses Dokument dient als Leitfaden für die Entwicklungsteams, um die spezifischen Anforderungen, Funktionen und Qualitäten festzulegen, die in den Smart-Spiegel integriert werden sollen, um ein hochwertiges und benutzerfreundliches Produkt zu liefern.

\section{Zielsetzung}

Ziel des Projekts ist es, einen Smart-Spiegel zu entwickeln, der die tägliche Routine der Benutzer verbessert, indem er relevante Informationen und Funktionen in einem intuitiven und benutzerfreundlichen Format bereitstellt. Der Smart-Spiegel soll dazu beitragen, den Alltag der Benutzer effizienter zu gestalten, indem er wichtige Informationen wie Wettervorhersagen, Verkehrsinformationen und Kalenderereignisse auf einen Blick zugänglich macht.  

\section{Zielgruppe}
\begin{itemize}
    \item Technikbegeisterte: Personen, die an innovativen Technologien interessiert sind und gerne neue Gadgets und intelligente Geräte ausprobieren.
    \item Smart-Home-Enthusiasten: Menschen, die bereits Smart-Home-Systeme verwenden oder ihr Zuhause mit vernetzten Geräten ausstatten möchten und nach neuen Möglichkeiten suchen, ihr Zuhause zu automatisieren und zu verbessern.
    \item Ältere Menschen und Menschen mit besonderen Bedürfnissen: Personen, die von Assistenzfunktionen wie Gesichtserkennung, Sprachsteuerung und Notfallbenachrichtigungen profitieren könnten, um ihre Unabhängigkeit zu fördern und ihre Sicherheit zu verbessern.
\end{itemize}

\section{Funktionale Anforderungen}
\begin{enumerate}[label=\textbf{FA\arabic*:}]
    \item Anzeige von Informationen: Der Smart-Spiegel muss in der Lage sein, die aktuelle Wetterlage eines Ortes und Kalenderereignisse anzuzeigen.
    \item Personalisierungsoptionen: Der Benutzer soll in der Lage sein, mindestens 10 Widgets und  5 ausgewählte Kalenderereignisse nach seinen persönlichen Vorlieben anzupassen.
    \item Interaktive Funktionen: Der Smart-Spiegel muss interaktive Funktionen bieten. Dazu hat der Benutzer die Möglichkeit, mit einer mobilen Kontroll-Applikation oder durch Sprachsteuerung das Interface zu navigieren.
    \item Gesichtserkennung: Möglichkeit zur Gesichtserkennung für die Identifizierung von Benutzern und Anpassung der angezeigten Inhalte entsprechend den individuellen Präferenzen mithilfe von künstlicher Intelligenz.
    \item Datenschutz und Sicherheit: Implementierung von Sicherheitsfunktionen, um die Privatsphäre der Benutzer zu schützen, wie die Verschlüsselung von Datenübertragungen und Optionen zur Deaktivierung von der Gesichtserkennung.
    \item Automatisches Einschalten: Das Display des Spiegels wird aktiviert, sobald sich die Person dem Spiegel nähert.
\end{enumerate}

\section{Nicht-funktionale Anforderungen}
\begin{enumerate}[label=\textbf{NFA\arabic*:}]
    \item Sprache: Der Displaytext und alle Informationen sollen in Deutsch angezeigt werden.
    \item Zuverlässigkeit: Der Smart-Spiegel soll stabil und zuverlässig funktionieren. Der Spiegel soll maximal 3x im Monat abstürzen oder ausfallen.
    \item Performance: Das System soll eine schnelle Reaktionszeit auf Benutzereingaben bieten . Das System soll innerhalb einer Sekunde nach einer Benutzereingabe die Aktion ausführen.
    \item Barrierefreiheit: Das System soll mindestens 95 Prozent der persönlichen Benutzerdaten durch eine Ende-zu-Ende-Verschlüsselung schützen und eine Datenschutzrichtlinie enthalten, die den geltenden Datenschutzgesetzen entspricht.
    \item Einhaltung gesetzlicher Datenschutzvorschriften: Das System muss die Einhaltung der Datenschutz-Grundverordnung (DSGVO) der EU und anderer relevanter lokaler Datenschutzgesetze nachweisen. Ein Compliance-Bericht muss jährlich erstellt und von einer externen Datenschutzbehörde oder einem zertifizierten Datenschutzbeauftragten geprüft werden.
    \item Wartbarkeit: Der Code des Systems sollte eine Testabdeckung von mindestens 80 Prozent erreichen und es sollte eine Dokumentation für Entwickler bereitgestellt werden, die die Einrichtung, Wartung und Aktualisierung des Systems detailliert beschreibt.
\end{enumerate}

\section{Anforderungen an die Benutzeroberfläche}
\begin{enumerate}[label=\textbf{B\arabic*:}]
    \item Die Benutzeroberfläche soll intuitiv und für Benutzer aller Erfahrungsstufen zugänglich sein, mit einer klaren und logischen Struktur, unterstützt durch leicht erkennbare Symbole und eine einheitliche Designsprache. Usability-Tests sollen bestätigen, dass mindestens 80 Prozent der Testpersonen die Handhabung als einfach oder sehr einfach bewerten.
    \item Es muss eine klare und einfache Navigation vorhanden sein, die es Benutzern ermöglicht, mühelos zwischen verschiedenen Funktionen und Bildschirmen zu wechseln. Die Effektivität der Navigation wird durch Benutzertests mit einer Erfolgsquote von mindestens 80 Prozent bei der Aufgabenbewältigung gemessen.
    \item Die Benutzeroberfläche soll personalisierbar sein, sodass Benutzer ihre bevorzugten Einstellungen und Anzeigen individuell anpassen können. Die Anpassungsfähigkeit wird durch die Bereitstellung von Optionen für die Auswahl und Anordnung von Widgets sowie die Personalisierung von Farbschemata und Schriftgrößen sichergestellt.
\end{enumerate}

\section{Budget}
Das Budget für die Materialien liegt bei 100 €.

\section{Abnahmekriterien}
Das Projekt Spiegel AI wird als abgeschlossen betrachtet, wenn folgende Kriterien erfüllt sind:

\begin{enumerate}
    \item Alle funktionalen Anforderungen gemäß Abschnitt 4 wurden erfolgreich implementiert und getestet.
    \item Das System wurde auf Zuverlässigkeit und Performance getestet und erfüllt die Anforderungen gemäß Abschnitt 5.
    \item Die Sicherheitsmaßnahmen des Smart-Spiegels, einschließlich Datenschutz und Datenschutz, wurden gemäß Abschnitt 5 implementiert und getestet.
    \item Die Benutzeroberfläche des Smart-Spiegels erfüllt die Anforderungen gemäß Abschnitt 6 und bietet eine intuitive und ansprechende Benutzererfahrung.
    \item Das System wurde einem umfassenden Benutzertest unterzogen, bei dem technikaffine Benutzer mit verschiedenen Hintergründen die Benutzerfreundlichkeit und Funktionalität bewertet haben. Das Feedback wurde berücksichtigt und etwaige Mängel wurden behoben.
    \item Alle dokumentierten Änderungen und Anpassungen, einschließlich eventueller Iterationen, wurden gemäß Abschnitt 9 in die endgültige Version des Lastenhefts aufgenommen.
\end{enumerate}

Die Zustimmung zur Abnahme erfolgt durch das Projektmanagement oder den Auftraggeber nach erfolgreicher Überprüfung der oben genannten Kriterien.


\section{Änderungshistorie}
\begin{tabular}{|c|c|c|l|}
    \hline
    Version & Datum & Autor & Beschreibung \\
    \hline
    1.0 & 27.03.2024 & David Vollmer & erstes Exemplar des Lastenhefts \\
    \hline
    2.0 & 28.03.2024 & Leon Kranner & Redundanzen behoben. Abschnitt 1 bis 3 überarbeitet \\
    \hline
    2.1 & 01.04.2024 & David Vollmer & Anpassung des Budgets \\
    \hline
    3.0 & 07.04.2024 & David Vollmer & Hinzufügen der Namen der Teilnehmer \\
    \hline
    4.0 & 09.04.2024 & Leon Kranner & Quantifizierung bei einigen Anforderungen ergänzt \\
    \hline
    5.0 & 10.04.2024 & Marco Kuner & Quantifizierung der Anforderungen vervollständigt \\
    \hline
    5.1 & 14.04.2024 & David Vollmer & Anpassung der Abnahmekriterien \\
    \hline
\end{tabular}

\end{document}
