\documentclass{article}
\usepackage{enumitem}

\title{Lastenheft für das Projekt Spiegel AI}
\author{Gruppe 5}
\date{\today}

% TODO: Anforderungen an die Benutzeroberfläche: mögliche Redundanz entfernen, evtl. diese Anforderungen nur in funktionale Anforderungen angeben.
% TODO: Tatsächliches Budget angeben.
% TODO: Einführung, Zielsetzung und Zielgruppe überdenken, evtl. umformulieren !DONE!
% TODO: Bestehende Anforderungen noch mal überprüfen, evtl. neue Anforderungen hinzufügen.
% Bei neuer Version des Lastenhefts bitte die Änderungshistorie anpassen. Danke :)

\begin{document}
\maketitle

\section{Einführung}
Dieses Lastenheft beschreibt die Anforderungen und Spezifikationen für die Entwicklung eines Smart-Spiegels. Der Smart-Spiegel ist ein innovatives Gerät, das traditionelle Spiegeltechnologie mit moderner Informationstechnologie kombiniert, um dem Benutzer eine Vielzahl von Funktionen und Diensten zu bieten. Entwickelt wird dieser Smart-Spiegel, um den täglichen Lebensstil der Benutzer zu bereichern, indem er relevante Informationen und Dienste direkt während des Morgenrituals im Badezimmer oder an anderen strategischen Standorten im Haushalt zur Verfügung stellt. Dieses Dokument dient als Leitfaden für die Entwicklungsteams, um die spezifischen Anforderungen, Funktionen und Qualitäten festzulegen, die in den Smart-Spiegel integriert werden sollen, um ein hochwertiges und benutzerfreundliches Produkt zu liefern.

\section{Zielsetzung}

Ziel des Projekts ist es, einen Smart-Spiegel zu entwickeln, der die tägliche Routine der Benutzer verbessert, indem er relevante Informationen und Funktionen in einem intuitiven und benutzerfreundlichen Format bereitstellt. Der Smart-Spiegel soll dazu beitragen, den Alltag der Benutzer effizienter zu gestalten, indem er wichtige Informationen wie Wettervorhersagen, Verkehrsinformationen und Kalenderereignisse auf einen Blick zugänglich macht.  

\section{Zielgruppe}
\begin{itemize}
    \item Technikbegeisterte: Personen, die an innovativen Technologien interessiert sind und gerne neue Gadgets und intelligente Geräte ausprobieren.
    \item Smart-Home-Enthusiasten: Menschen, die bereits Smart-Home-Systeme verwenden oder ihr Zuhause mit vernetzten Geräten ausstatten möchten und nach neuen Möglichkeiten suchen, ihr Zuhause zu automatisieren und zu verbessern.
    \item Ältere Menschen und Menschen mit besonderen Bedürfnissen: Personen, die von Assistenzfunktionen wie Gesichtserkennung, Sprachsteuerung und Notfallbenachrichtigungen profitieren könnten, um ihre Unabhängigkeit zu fördern und ihre Sicherheit zu verbessern.
\end{itemize}

\section{Funktionale Anforderungen}
\begin{enumerate}[label=\textbf{FA\arabic*:}]
    \item Anzeige von Informationen: Der Smart-Spiegel muss in der Lage sein, die aktuelle Wetterlage eines Ortes und Kalenderereignisse anzuzeigen.
    \item Personalisierungsoptionen: Der Benutzer soll in der Lage sein, Widgets und ausgewählte Kalenderereignisse nach seinen persönlichen Vorlieben anzupassen.
    \item Interaktive Funktionen: Der Smart-Spiegel muss interaktive Funktionen bieten. Dazu hat der Benutzer die Möglichkeit, mit einer mobilen Kontroll-Applikation oder durch Sprachsteuerung das Interface zu navigieren.
    \item Gesichtserkennung: Möglichkeit zur Gesichtserkennung für die Identifizierung von Benutzern und Anpassung der angezeigten Inhalte entsprechend den individuellen Präferenzen mithilfe von künstlicher Intelligenz.
    \item Datenschutz und Sicherheit: Implementierung von Sicherheitsfunktionen, um die Privatsphäre der Benutzer zu schützen, wie die Verschlüsselung von Datenübertragungen und Optionen zur Deaktivierung von der Gesichtserkennung.
    \item Automatisches Einschalten: Das Display des Spiegels wird aktiviert, sobald sich die Person dem Spiegel nähert.
\end{enumerate}

\section{Nicht-funktionale Anforderungen}
\begin{enumerate}[label=\textbf{NFA\arabic*:}]
    \item Zuverlässigkeit: Der Smart-Spiegel soll stabil und zuverlässig funktionieren, ohne häufige Ausfälle oder Systemabstürze.
    \item Performance: Das System soll eine schnelle Reaktionszeit auf Benutzereingaben bieten und Inhalte flüssig und ohne Verzögerungen anzeigen.
    \item Skalierbarkeit: Die Architektur des Smart-Spiegels soll skalierbar sein, um zukünftiges Wachstum und die Integration neuer Funktionen zu unterstützen, ohne dass die Leistung beeinträchtigt wird.
    \item Sicherheit: Der Smart-Spiegel soll robuste Sicherheitsmaßnahmen implementieren, um Benutzerdaten vor unbefugtem Zugriff und Datenschutzverletzungen zu schützen.
    \item Barrierefreiheit: Das System soll barrierefrei sein und die Bedürfnisse von Benutzern mit verschiedenen Fähigkeiten berücksichtigen, z. B. durch alternative Bedienungsmöglichkeiten für Menschen mit motorischen Einschränkungen.
    \item Datenschutz: Es soll klare Richtlinien und Mechanismen geben, um die Privatsphäre der Benutzer zu schützen und sicherzustellen, dass persönliche Daten angemessen gesichert und verarbeitet werden.
    \item Wartbarkeit: Das System sollte wartungsfreundlich sein, mit klarem Code und dokumentierten Prozessen für die Fehlerbehebung und Aktualisierung.
    \item Energieeffizienz: Der Smart-Spiegel sollte energieeffizient sein und den Stromverbrauch optimieren, um die Umweltauswirkungen zu minimieren und die Betriebskosten zu senken.
\end{enumerate}

\section{Anforderungen an die Benutzeroberfläche}
\begin{enumerate}[label=\textbf{B\arabic*:}]
    \item Benutzeroberfläche muss ansprechend und intuitiv gestaltet sein, um eine reibungslose Interaktion mit dem Smart-Spiegel zu ermöglichen.
    \item Es soll eine klare Navigation vorhanden sein, die es dem Benutzer ermöglicht, mühelos zwischen verschiedenen Funktionen und Bildschirmen zu wechseln.
    \item Die Benutzeroberfläche soll personalisierbar sein, so dass Benutzer ihre eigenen Layouts und Einstellungen definieren können.
\end{enumerate}

\section{Anforderungen an die Datenbank}
\begin{enumerate}[label=\textbf{D\arabic*:}]
    \item Die Datenbank muss eine hohe Leistungsfähigkeit aufweisen, um eine schnelle Datenabfrage und -aktualisierung zu ermöglichen.
    \item Es sollte eine zuverlässige Datensicherung und Wiederherstellungsfunktion geben, um Datenverlust im Falle eines Systemausfalls zu verhindern.
    \item Die Datenbank muss skalierbar sein, um das Wachstum des Systems und die Zunahme der Datenmenge zu bewältigen.
    \item Datenschutz und Sicherheit sollten in der Datenbankimplementierung berücksichtigt werden, um den Schutz sensibler Benutzerdaten zu gewährleisten.
\end{enumerate}

\section{Budget}
Das Budget für die Materialien liegt bei 200 bis 300 €. (siehe CSV-Datei)

\section{Abnahmekriterien}
Das Projekt Spiegel AI wird als abgeschlossen betrachtet, wenn folgende Kriterien erfüllt sind:

\begin{enumerate}
    \item Alle funktionalen Anforderungen gemäß Abschnitt 4 wurden erfolgreich implementiert und getestet.
    \item Das System wurde auf Zuverlässigkeit und Performance getestet und erfüllt die Anforderungen gemäß Abschnitt 5.
    \item Die Sicherheitsmaßnahmen des Smart-Spiegels, einschließlich Datenschutz und Datenschutz, wurden gemäß Abschnitt 5 implementiert und getestet.
    \item Die Benutzeroberfläche des Smart-Spiegels erfüllt die Anforderungen gemäß Abschnitt 6 und bietet eine intuitive und ansprechende Benutzererfahrung.
    \item Die Datenbank des Systems erfüllt die Anforderungen gemäß Abschnitt 7 und gewährleistet eine zuverlässige und performante Speicherung sowie den Schutz sensibler Benutzerdaten.
    \item Das System wurde einem umfassenden Benutzertest unterzogen, bei dem technikaffine Benutzer mit verschiedenen Hintergründen die Benutzerfreundlichkeit und Funktionalität bewertet haben. Das Feedback wurde berücksichtigt und etwaige Mängel wurden behoben.
    \item Alle dokumentierten Änderungen und Anpassungen, einschließlich eventueller Iterationen, wurden gemäß Abschnitt 10 in die endgültige Version des Lastenhefts aufgenommen.
\end{enumerate}

Die Zustimmung zur Abnahme erfolgt durch das Projektmanagement oder den Auftraggeber nach erfolgreicher Überprüfung der oben genannten Kriterien.


\section{Änderungshistorie}
\begin{tabular}{|c|c|c|l|}
    \hline
    Version & Datum & Autor & Beschreibung \\
    \hline
    1.0 & 27.03.2024 & David Vollmer & erstes Exemplar des Lastenhefts \\
    \hline
    2.0 & 28.03.2024 & Leon Kranner & Redundanzen behoben. Abschnitt 1 bis 3 überarbeitet \\
    \hline
\end{tabular}

\end{document}
