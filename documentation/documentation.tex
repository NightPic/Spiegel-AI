\documentclass[a4paper,12pt]{report}
\usepackage[utf8]{inputenc}
\usepackage[T1]{fontenc}
\usepackage[ngerman]{babel}
\usepackage{graphicx}
\usepackage{tabularx}
\usepackage{longtable}
\usepackage{hyperref}
\usepackage{geometry}
\usepackage{titlesec}
\usepackage{xcolor}
\usepackage{helvet}
\usepackage{caption}
\usepackage{wrapfig}
\usepackage{textcomp}

\usepackage[backend=biber,style=ieee]{biblatex}
\addbibresource{literaturverzeichnis.bib}
\DeclareLanguageMapping{german}{german-ieee-custom}
\renewcommand{\familydefault}{\sfdefault}

\geometry{a4paper, margin=1in}

\titleformat{\chapter}[hang]{\bf\huge}{\thechapter \hspace{1em}\textcolor{blue}|}{0pt}{\huge\bf}

\hypersetup{
    colorlinks=true,
    linkcolor=black,
    filecolor=magenta,      
    urlcolor=cyan,
    citecolor=cyan,
}

\begin{document}

\begin{titlepage}
    \begin{flushleft}
        \includegraphics[height=1.2cm]{pictures/oth_logo.eps}
    \end{flushleft}
    \vspace{2cm}
    \begin{center}
        {\Huge\textbf{Spiegel AI}}\\[1cm]
        {\Large\textbf{Datenverarbeitung in der Technik}}\\[0.5cm]
        {\Large\textbf{Gruppe 5}}\\[1.5cm]
        \includegraphics[width=0.37\textwidth]{pictures/eggo_icon.jpeg}\\[1cm]
    \end{center}
    \vspace{2cm}
    \begin{flushleft}
        \begin{tabbing}
            \hspace{8cm} \= \hspace{4cm} \kill
            Leon Kranner \> \texttt{leon.kranner@st.oth-regensburg.de} \\
            Marco Kuner \> \texttt{marco.kuner@st.oth-regensburg.de} \\
            David Vollmer \> \texttt{david1.vollmer@st.oth-regensburg.de} \\
            Marcel Wagner \> \texttt{marcel.wagner@st.oth-regensburg.de} \\
        \end{tabbing}
    \end{flushleft}
    \vfill
    \begin{center}
        {\Large 15. Juli 2024}
    \end{center}
\end{titlepage}

\tableofcontents
\newpage

\chapter*{Einleitung}
\addcontentsline{toc}{chapter}{Einleitung}

In der Einleitung stellen wir das Projekt \textbf{Spiegel AI} vor. Wir beschreiben die Zielsetzung des Projekts, die Motivation und den allgemeinen Aufbau der Dokumentation. Zudem geben wir einen Überblick über die eingesetzte Hardware und Software sowie die geplanten Anwendungsbereiche.

\section*{Zielsetzung}
Beschreiben Sie hier die Zielsetzung des Projekts.

\section*{Motivation}
Erläutern Sie die Motivation hinter dem Projekt.

\section*{Überblick}
Geben Sie einen Überblick über die Struktur der Dokumentation.


\chapter{Rahmen}
Erarbeitet von: Leon Kranner und Marcel Wagner \\ \\
\noindent
In diesem Kapitel wird die Vorgehensweise für die Rahmenkonstruktion des Smart Mirrors beschrieben.

\section{Rahmen Konstruktion}
\begin{enumerate}
    \item \textbf{Planung:}
Zu Beginn der Planung haben wir die Abmessungen des Displays genau vermessen, um den benötigten Platz für die Hardwarekomponenten und das Gehäuse bestimmen zu können. Es war wichtig, genügend Raum für den Raspberry Pi, die Kamera für die Gesichtserkennung, den WLAN-Stick und eventuelle Kabelverbindungen einzuplanen.

Nachdem wir die Abmessungen des Displays und den Platzbedarf für die Hardware ermittelt hatten, konnten wir den Maßstab und die endgültige Größe des Rahmens festlegen. Unser Ziel war es, einen Rahmen zu konstruieren, der sowohl funktional als auch ästhetisch ansprechend ist.

Im nächsten Schritt besuchten wir den Baumarkt, um das passende Material für unseren Rahmen auszuwählen. Wir entschieden uns schließlich für Leimholz, da es robust und gut zu verarbeiten ist.

Ursprünglich war geplant, den Rahmen mit Schrauben zusammenzubauen. Nach weiteren Überlegungen und Tests fanden wir jedoch, dass die geschraubte Konstruktion unseren ästhetischen Ansprüchen nicht gerecht wurde. Daher entschieden wir uns, das Leimholz zu verwenden und die Teile zu verleimen. Diese Methode bot uns eine stabilere und sauberere Verbindung der Rahmenteile.
    \item \textbf{Zuschneiden des Leimholzes:}
    Zu Beginn der Konstruktion wurde das Leimholz auf die gewünschte Länge zugeschnitten. Dabei war es wichtig, präzise Maße zu verwenden, um sicherzustellen, dass alle Teile des Rahmens passgenau zueinander stehen. Für den Zuschnitt wurde eine Kreissäge verwendet, um gerade und saubere Schnitte zu erzielen.
    
    \item \textbf{Gehrungsschnitt der Kanten:}
    Um eine ästhetisch ansprechende und stabile Verbindung der Rahmenteile zu gewährleisten, wurden die Kanten des Holzes auf Gehrung geschnitten. Hierzu wurde ein Gehrungssägeblatt in einem Winkel von 45 Grad eingestellt. Dieser Schritt ist entscheidend, da die Gehrungsschnitte die Stoßkanten der Rahmenteile so ausrichten, dass sie sich nahtlos und formschlüssig verbinden lassen.
    
    \item \textbf{Verleimung der Rahmenteile:}
    Nach dem Gehrungsschnitt wurden die Kanten mit Holzleim bestrichen. Es ist wichtig, den Leim gleichmäßig aufzutragen, um eine vollständige und feste Verbindung zu gewährleisten. Die Rahmenteile wurden dann sorgfältig zusammengesetzt, wobei darauf geachtet wurde, dass die Gehrungsschnitte exakt aufeinanderpassen. \\ \\
Zur Sicherstellung einer stabilen Verbindung wurden die verleimten Rahmenteile mit Schraubzwingen fixiert. Die Schraubzwingen wurden gleichmäßig verteilt angezogen, um den Druck auf alle Teile des Rahmens zu verteilen und Verformungen zu vermeiden. Die Rahmenkonstruktion wurde dann für die empfohlene Zeitspanne in den Schraubzwingen belassen, um eine vollständige Aushärtung des Leims zu gewährleisten. In der nachfolgenden Abbildung kann dar Spiegelrahmen entnommen werden.

\begin{figure}[h]
    \centering
    \includegraphics[width=0.5\textwidth]{pictures/Rahmen_geleimt.jpg}
  \captionsetup{justification=centering, labelformat=simple, singlelinecheck=false}
    \caption[Rahmen wurde geleimt]{Rahmen wurde geleimt\\ Quelle: eigene Darstellung}
\end{figure}
    
    \item \textbf{Befestigung der Halteleiste:}
    Nach der Aushärtung des Leims wurde eine Halteleiste angebracht. Diese Leiste dient dazu, den Spiegel sicher im Rahmen zu fixieren. Die Halteleiste wurde präzise vermessen und zugeschnitten, um optimal in den Rahmen zu passen. Sie wurde mit Holzleim und zusätzlichen Schrauben befestigt, um eine dauerhafte und sichere Fixierung zu gewährleisten.
\end{enumerate}
\begin{figure}[h]
    \centering
    \includegraphics[width=0.3\textwidth]{pictures/Rahmen_fertig.jpg}
  \captionsetup{justification=centering, labelformat=simple, singlelinecheck=false}
    \caption[Fertiggestellter Rahmen]{Fertiggestellter Rahmen\\ Quelle: eigene Darstellung}
\end{figure}
\newpage

\section{Spiegel}
\begin{enumerate}
    \item \textbf{Auswahl des Materials:}
Ursprünglich hatten wir geplant, eine Glasscheibe für den Spiegel des Smart Mirrors zu verwenden. Da jedoch für unsere Zwecke eine Glasscheibe nicht erforderlich war und wir bereits eine Plexiglasscheibe zur Verfügung hatten, entschieden wir uns, diese zu verwenden.

    \item \textbf{Zuschnitt und Vorbereitung:}
Zunächst haben wir anhand des Rahmens abgemessen, wie groß der Spiegel sein muss, und eine Toleranz von 1 cm festgelegt. Mit Hilfe einer Stichsäge haben wir das Plexiglas auf Maß geschnitten. Anschließend haben wir die Kanten mit einer Feile entgratet und mit Schleifpapier die unebenen Stellen beseitigt.
    
    \item \textbf{Bohrungen und Befestigung:}
Um das Plexiglas später am Rahmen befestigen zu können, haben wir an den Seiten des Plexiglases Bohrungslöcher gebohrt.
    
    \item \textbf{Anbringen der Spiegelfolie:}
    Zunächst haben wir eine Spiegelglasfolie auf das Plexiglas aufgebracht. Beim Testen mit dem Bildschirm stellten wir jedoch fest, dass die Helligkeit des Displays nicht ausreichte, um den Bildschirm durch das Plexiglas und die Spiegelglasfolie zu erkennen. Aufgrund der unzureichenden Helligkeit des Displays haben wir uns für eine Sonnenschutzfolie entschieden. Diese Folie spiegelt immer noch, ermöglichte dennoch eine bessere Sicht.

\end{enumerate}
\section{Aufbau}
\begin{enumerate}
    \item \textbf{Befestigung des Plexiglases:}
Nachdem die geeignete Folie angebracht war, haben wir das Plexiglas mit Schrauben fixiert. Diese Maßnahme gewährleistete eine stabile und sichere Befestigung.

    \item \textbf{Montage des Bildschirms:}
Der erste Ansatz zur Befestigung des Bildschirms bestand darin, ein Draht von der linken zur rechten Seite des Spiegels zu spannen. Diese Lösung fixiert den Bildschirm in der gewünschten Position und verhinderte ein Verrutschen. Aufgrund der Anfälligkeit für Spielräume und der Tatsache, dass der Bildschirm nicht immer fest mit dem Plexiglas verbunden ist, entschieden wir uns jedoch, den Bildschirm mit eng anliegenden Haltern am Bilderrahmen zu fixieren. Diese Konstruktion gewährleistet eine klare und stabile Befestigung des Bildschirminhalts durch die Spiegelfolie.
    
    \item \textbf{Integration der Elektronik:}
An der Seite des Rahmens haben wir den Raspberry Pi befestigt und den Bildschirm angeschlossen. Der Raspberry Pi steuert die gesamte Funktionalität des Smart Mirrors. In der nachfolgenden Abbildung kann der befestigte Raspberry Pi an dem Spiegerrahmen entnommen werden.
\begin{figure}[h]
    \centering
    \includegraphics[width=0.3\textwidth]{pictures/Raspberry_pi_Befestigung.jpg}
  \captionsetup{justification=centering, labelformat=simple, singlelinecheck=false}
    \caption[Raspberry Pi am Rahmen befestigt]{Raspberry Pi am Rahmen befestigt\\ Quelle: eigene Darstellung}
\end{figure}
\newpage
    \item \textbf{Zusätzliche Komponenten:}
    Wir haben zudem eine Kamera für die Gesichtserkennung (Eggo AI) angeschlossen. Diese Kamera ermöglicht personalisierte Funktionen und verbessert die Benutzererfahrung. Zusätzlich haben wir einen WLAN-Stick integriert, damit die Widgets Daten aus dem Internet abrufen und mit der App kommunizieren können

\end{enumerate}

Der nachfolgende Bild zeigt den fertigen Smart Mirror.
\begin{figure}[h]
    \centering
    \includegraphics[width=0.3\textwidth]{pictures/Spiegel_AI.jpg}
  \captionsetup{justification=centering, labelformat=simple, singlelinecheck=false}
    \caption[Spiegel AI]{Spiegel AI\\ Quelle: eigene Darstellung}
\end{figure}

\chapter{Hardware}

In diesem Kapitel beschreiben wir die Hardware-Komponenten, die für das Projekt \textbf{Spiegel AI} verwendet wurden. Wir gehen auf die Auswahlkriterien, die Installation und die Konfiguration der Hardware ein.

\section{Komponenten}
Beschreiben Sie die einzelnen Hardware-Komponenten und deren Spezifikationen.

\section{Auswahlkriterien}
Erläutern Sie die Kriterien, nach denen die Hardware ausgewählt wurde.

\section{Installation}
Beschreiben Sie den Installationsprozess der Hardware.

\section{Konfiguration}
Erläutern Sie die Konfiguration der Hardware-Komponenten.


\chapter{Display}

In diesem Kapitel gehen wir auf das Display ein, das im \textbf{Spiegel AI} Projekt verwendet wird. Wir beschreiben die Spezifikationen, die Installation und die Anpassungen, die vorgenommen wurden.

\section{Spezifikationen}
Beschreiben Sie die technischen Spezifikationen des Displays.

\section{Installation}
Erläutern Sie den Prozess der Installation des Displays.

\section{Anpassungen}
Beschreiben Sie etwaige Anpassungen oder Modifikationen am Display.

\subsection{eventuell HTMl seite und Aufbau oder in Installation}
\subsection{Widget 1}
\subsection{Widget 2}
\subsection{Widget 3}
\subsection{Widget 4}
\subsection{Uhr Widget}
Erarbeitet von: Marcel Wagner \\ \\
\noindent
Die Implementierung des Uhrzeit Widgets für den Smart Mirror ist ein wichtiger Schritt zur Verbesserung der Funktionalität und Benutzerfreundlichkeit des Geräts. Ziel dieses Widgets ist es, die aktuelle Uhrzeit exakt und zuverlässig anzuzeigen. Wobei die Anzeige in Echtzeit aktualisiert werden muss, um stets die genaue Uhrzeit widerzuspiegeln.\\ \\
Die Implementierung dieses Widgets basierte auf der Nutzung von JavaScript zur Echtzeitaktualisierung der Uhrzeit und HTML zur Einbettung des Widgets in die Benutzeroberfläche des Smart Mirrors. Desweiteren wurde CSS benutzt um das Widget zu formatieren. Die JavaScript Funktion sorgt dafür, dass die Uhrzeit jede Sekunde aktualisiert wird, während das HTML Dokument die Struktur definiert. Abschließend definiert die CSS Datei das Styling des Widgets. \\ \\
\noindent
Während der Entwicklung des Widgets traten mehrere Herausforderungen auf. Eine der größten Herausforderungen bestand darin, sicherzustellen, dass die Uhrzeit in Echtzeit und ohne Verzögerung aktualisiert wird. Dies war besonders wichtig, um die Genauigkeit der angezeigten Zeit zu gewährleisten. Die Verwendung der 'setTimeout' Funktion in JavaScript ermöglicht eine wiederholte Ausführung der Aktualisierungsfunktion in einem festgelegten Intervall von einer Sekunde, wodurch eine kontinuierliche und genaue Aktualisierung der Uhrzeit sichergestellt wurde.
Eine weitere Herausforderung war die exakte Zeitanzeige, insbesondere hierbei ist wichtig die Erwähnung der Formatierung der Uhrzeit, um sicherzustellen, dass Stunden, Minuten und Sekunden stets zweistellig angezeigt werden. Durch die Verwendung der 'padStart' Methode konnten die Zahlen auf eine konstante Länge von zwei Stellen gebracht werden, indem bei Bedarf führende Nullen hinzugefügt werden. Dies gewährleistete eine konsistente und gut lesbare Anzeige.\\ \\
\noindent
Die Implementierung des Uhrzeit Widgets verlief erfolgreich und erfüllt die gestellten Anforderungen. Die Uhrzeit wird zuverlässig und exakt in Echtzeit angezeigt. Das Widget integriert sich nahtlos in die Benutzeroberfläche des Smart Mirrors und bietet eine klare und gut lesbare Darstellung der aktuellen Uhrzeit.
Insgesamt stellt das Uhrzeit Widget eine wesentliche Funktionalität des Smart Mirrors dar. Der nachfolgenden Abbildung 1 kann das Implementierte Uhrzeit Widget auf der HTML Seite entnommen werden.

\begin{figure}[h]
    \centering
    \includegraphics[width=0.4\textwidth]{pictures/time_widget.png}
  \captionsetup{justification=centering, labelformat=simple, singlelinecheck=false}
    \caption{ Uhrzeit Widget\\ Quelle: eigene Darstellung}
\end{figure}

\subsection{Verkehrsinformation}
Erarbeitet von: Marcel Wagner \\ \\
\noindent
Die Implementierung des Stau-Widgets auf dem Smart Mirror stellt einen wichtigen Schritt dar, um den Nutzern eine umfassende und zuverlässige Quelle für aktuelle Verkehrsinformationen zur Verfügung zu stellen. Das Widget wurde speziell entwickelt, um eine Echtzeitübersicht über die Verkehrslage in Regensburg zu bieten, was insbesondere für Pendler und Reisende von großem Nutzen ist. Durch die Verwendung von JavaScript wurde eine nahtlose Integration mit der OpenStreetMap Overpass API realisiert, die als zuverlässige Datenquelle für Verkehrsdaten dient. \\ \\
\noindent
Die Strategie hinter der Implementierung war zweigleisig: Zum einen wurde eine sofortige Aktualisierung der Verkehrsinformationen beim Laden der Seite implementiert, um den Nutzern bei jedem Besuch des Smart Mirrors die aktuellsten Daten bereitzustellen. Zum anderen erfolgt eine regelmäßige automatische Aktualisierung alle fünf Minuten, um sicherzustellen, dass die angezeigten Informationen kontinuierlich aktuell gehalten werden. Dieser Ansatz gewährleistet eine hohe Aktualität und Relevanz der bereitgestellten Verkehrsinformationen. \\ \\
\noindent
Während der Entwicklung wurden mehrere Herausforderungen gemeistert, darunter die robuste Fehlerbehandlung, um sicherzustellen, dass Netzwerkprobleme oder API Ausfälle die Funktionalität des Widgets nicht beeinträchtigen. Ein besonderes Augenmerk lag auf der Gewährleistung einer stabilen und zuverlässigen Datenaktualisierung, die für eine nahtlose Benutzererfahrung entscheidend ist. \\ \\
\noindent
Das Verkehrs Widget präsentiert die Verkehrslage in einer klaren und intuitiven Benutzeroberfläche. Es informiert die Nutzer klar verständlich darüber, ob derzeit ein Stau vorliegt oder nicht, und bietet gegebenenfalls zusätzliche Informationen über Verkehrshindernisse oder Verkehrswarnungen. Diese klare visuelle Darstellung hilft den Nutzern, schnell zu erfassen, wie die aktuelle Verkehrssituation ihre geplante Route beeinflussen ist. \\ \\
\noindent
Insgesamt trägt das Verkehrs Widget erheblich zur Funktionalität und Benutzerfreundlichkeit des Smart Mirrors bei. Es bietet eine unverzichtbare Informationsquelle für die tägliche Routenplanung und unterstützt die Nutzer dabei, ihre Fahrtzeiten effizient zu optimieren. Das Implementierte Verkehrsinformationen Widget kann der nachfolgenden Abbildung entnommen werden. Diese Abbildung zeigt den Fall, dass aktuell gerade kein Stau  in den Straßen von Regensburg sind.

\begin{figure}[h]
    \centering
    \includegraphics[width=0.4\textwidth]{pictures/traffic_widget.png}
  \captionsetup{justification=centering, labelformat=simple, singlelinecheck=false}
    \caption{Verkehrsinformations Widget\\ Quelle: eigene Darstellung}
\end{figure}

\subsection{Schlagzeilen}
Erarbeitet von: Marcel Wagner \\ \\
\noindent
Die Implementierung des Nachrichten Widgets für den Smart Mirror stellt einen wichtigen Schritt dar, um den Nutzern eine aktuelle und relevante Informationsquelle direkt auf seinem Smart Mirror zur Verfügung zu stellen. Das Widget wurde in JavaScript entwickelt und verwendet die 'RSS2JSON-API', um die neuesten Nachrichtenartikel eines ausgewählten RSS Feeds abzurufen und auf dem Smart Mirror anzuzeigen. Dies ermöglicht eine dynamische und automatische Aktualisierung der Nachrichteninhalte, sobald der Nutzer den Spiegel nutzt. \\ \\
\noindent
Ein zentrales Element der Implementierung ist die Verwendung des 'DOMContentLoaded' Events, das sicherstellt, dass das Widget erst aktiv wird, nachdem die gesamte Seite vollständig geladen ist. Dadurch wird sichergestellt, dass alle notwendigen Ressourcen und Elemente bereitstehen, bevor die Datenabfrage und die Darstellung der Nachrichten beginnen. \\ \\
\noindent
Die Funktionalität des Widgets umfasst die Asynchronität der Datenabfrage über die Fetch API, die die RSS Feeds von Nachrichtenquellen in ein JSON Format umwandelt, das vom JavaScript Code weiterverarbeitet werden kann. Dies ermöglicht eine schnelle und effiziente Bereitstellung der neuesten Nachrichteninhalte direkt auf dem Smart Mirror, ohne dass der Nutzer zusätzliche Schritte unternehmen muss, um sich auf dem Laufenden zu halten. \\ \\
\noindent
Eine besondere Herausforderung während der Implementierung war die unterschiedliche Verfügbarkeit von RSS Feeds bei verschiedenen Nachrichtenseiten. Viele führende Nachrichtenagenturen und Zeitungen bieten zwar RSS Feeds an, einige jedoch nicht oder beschränken den Zugang zu ihren Inhalten über diese Schnittstelle. Dies erforderte eine sorgfältige Auswahl geeigneter RSS Feeds, die eine kontinuierliche und zuverlässige Datenversorgung gewährleisten konnten. Die Ausgegeben Nachrichten dieses Widget entspannen der Frankfurter Allgemeinen Zeitung \\ \\
\noindent
Um die Benutzerfreundlichkeit zu maximieren, wurde die Benutzeroberfläche des Widgets bewusst einfach und intuitiv gestaltet. Die angezeigten Nachrichten werden in einer geordneten Liste präsentiert. \\ \\
\noindent
Zusammenfassend bietet das Nachrichten Widget einen bedeutenden Mehrwert für den Smart Mirror, indem es den Nutzern eine einfache und effektive Möglichkeit bietet, sich über aktuelle Ereignisse zu informieren. Die Implementierung war erfolgreich in Bezug auf die gesetzten Ziele. Der Nachfolgenden Abbildung kann das implementierte Widget auf dem Smart Mirror entnommen werden.

\begin{figure}[h]
    \centering
    \includegraphics[width=0.4\textwidth]{pictures/news_widget.png}
  \captionsetup{justification=centering, labelformat=simple, singlelinecheck=false}
    \caption{Verkehrsinformations Widget\\ Quelle: eigene Darstellung}
\end{figure}

\subsection{Tankstellen}
\subsection{Test Verfahren}
Erarbeitet von: Leon Kranner und Marcel Wagner \\ \\
\noindent




\chapter{Spiegel AI Remote}

In diesem Kapitel wird das \textbf{Spiegel AI Remote} beschrieben. Wir gehen auf die Funktionen, die Architektur und die Implementierung der Fernsteuerung ein.

\section{Funktionen}
Beschreiben Sie die Hauptfunktionen des \textbf{Spiegel AI Remote}.

\section{Architektur}
Erläutern Sie die Architektur der Fernsteuerung.

\section{Implementierung}
Beschreiben Sie die Implementierung der Fernsteuerung und die verwendeten Technologien.


\chapter{Gesichtserkennung}

\begin{abstract}
Diese Dokumentation beschreibt die Entwicklung einer Gesichtserkennungslösung für einen Smart Mirror. Das Projekt wurde im Rahmen eines Vier-Personen-Teams durchgeführt und beinhaltet die Recherche, Implementierung und Optimierung verschiedener Gesichtserkennungstechnologien. Besonderes Augenmerk liegt auf der genauen Protokollierung der Entscheidungsprozesse und der technischen Herausforderungen.
\end{abstract}



\section{Einleitung}

\subsection{Projektziel}
Das Hauptziel dieses Projekts war die Entwicklung einer fortschrittlichen Gesichtserkennungslösung für einen Smart Mirror. Dieser Smart Mirror sollte in der Lage sein, Benutzer anhand ihrer Gesichter zu erkennen und personalisierte Informationen anzuzeigen. Die Gesichtserkennung sollte zuverlässig unter verschiedenen Bedingungen wie wechselnden Lichtverhältnissen und unterschiedlichen Gesichtswinkeln funktionieren.

\subsection{Bedeutung der Gesichtserkennung}
Gesichtserkennungstechnologien haben in den letzten Jahren erheblich an Bedeutung gewonnen. Sie finden Anwendungen in zahlreichen Bereichen wie Sicherheit, wo sie zur Zugangskontrolle und Überwachung eingesetzt werden, in der Personalisierung, wo sie individuelle Benutzererlebnisse ermöglichen, und in der Benutzerfreundlichkeit, da sie eine nahtlose und intuitive Interaktion mit technischen Geräten bieten. Die Entwicklung einer zuverlässigen Gesichtserkennung für den Smart Mirror ist also zwingend erforderlich für eine reibungslose Kundenerfahrung.



\section{Recherche und Anfangsphase}

\subsection{Grundlagen der Gesichtserkennung}
Gesichtserkennung ist ein Bereich der Computer Vision, der sich mit der Identifikation von Individuen anhand ihrer Gesichtszüge beschäftigt. Dies geschieht durch den Einsatz von Algorithmen, die charakteristische Merkmale eines Gesichts extrahieren und analysieren. Traditionelle Methoden der Gesichtserkennung umfassen Ansätze wie die Verwendung von Haar-Cascades, während moderne Methoden oft auf tiefen neuronalen Netzwerken basieren.

\subsection{Vergleich von Methoden}
In der Anfangsphase des Projekts wurde zunächst die Möglichkeit in Betracht gezogen, ein eigenes Deep-Learning-Modell für die Gesichtserkennung zu trainieren. Nach weiterer Recherche und intensiver Absprache mit Kommilitonen aus dem KI-Studiengang wurde jedoch festgestellt, dass das Training eines eigenen Modells aufgrund des hohen Zeitaufwands und der benötigten Rechenressourcen nicht praktikabel wäre.

Daraufhin wurde eine umfassende Recherche zu verschiedenen existierenden Methoden der Gesichtserkennung durchgeführt. Dabei wurden herkömmliche Ansätze wie Haar-Cascades und moderne Ansätze wie Deep Learning verglichen. Haar-Cascades, die auf der Viola-Jones-Methode basieren, bieten den Vorteil einer schnellen Berechnung und einfachen Implementierung, sind jedoch in ihrer Genauigkeit und Robustheit begrenzt. Im Gegensatz dazu bieten moderne Deep-Learning-Ansätze, wie sie in der Dlib-Bibliothek verwendet werden, eine höhere Genauigkeit und Robustheit, erfordern jedoch mehr Rechenleistung und sind komplexer in der Implementierung.



\section{Erste Implementierung mit Haar-Cascades}

\subsection{Haar-Cascade-Ansatz}
Aufgrund des begrenzten Speichers unseres Raspberry Pi wurde zunächst der Haar-Cascade-Ansatz gewählt, da dieser deutlich weniger komplex aufgebaut ist und weniger Rechenleistung erfordert. Haar-Cascades basieren auf der Viola-Jones-Methode, die einen robusten Algorithmus zur Gesichtserkennung darstellt. 

Die Haar-Cascade-Methode verwendet eine Kaskade von sogenannten Haar-ähnlichen Merkmalen\cite{haar_quelle}. Eine Kaskade in diesem Kontext bedeutet eine Abfolge von Klassifikatoren, die nacheinander angewendet werden, um die Erkennungsgenauigkeit zu erhöhen und gleichzeitig die Rechenleistung zu optimieren. Diese Merkmale sind einfache Muster, die in unterschiedlichen Größen und Positionen auf das Bild angewendet werden, um Kontraste zu erkennen, die typisch für Gesichtszüge sind. 

\begin{figure}[h!]
    \centering
    \includegraphics[width=0.5\textwidth]{pictures/haarcascades.jpg}
    \caption{Beispiel verschiedener Haar-ähnlicher Merkmale}
    \label{fig:haar_cascade_example}
    \cite{haar_cascade_example}
\end{figure}

Ein integrales Bild wird verwendet, um diese Merkmale effizient zu berechnen. Die Viola-Jones-Methode besteht aus mehreren Hauptkomponenten:

\textbf{Merkmalserkennung}: Haar-ähnliche Merkmale bestehen aus einfachen rechteckigen Bereichen, die Intensitätsunterschiede innerhalb des Bildes messen. Es gibt drei Arten von Haar-ähnlichen Merkmalen: Kantenmerkmale, Linienmerkmale und vierrechteckige Merkmale. Diese Merkmale helfen dabei, grundlegende Strukturen wie Kanten, Linien und Ecken zu erfassen, die in Gesichtern häufig vorkommen.

\textbf{Integralbild}: Das Integralbild ist eine Datenstruktur, die verwendet wird, um die Berechnung von Rechteckmerkmalen in konstanter Zeit zu ermöglichen. Dies wird erreicht, indem für jedes Pixel die Summe aller Pixelwerte oben und links davon berechnet wird. Dadurch kann jedes Rechteckmerkmal durch wenige Zugriffe auf das Integralbild effizient berechnet werden.

\textbf{Adaboost-Training}: Um die Merkmale zu einem starken Klassifikator zu kombinieren, wird der Adaboost-Algorithmus verwendet. Adaboost ist eine Methode des maschinellen Lernens, die eine große Anzahl schwacher Klassifikatoren zu einem starken Klassifikator kombiniert. Während des Trainingsprozesses werden die wichtigsten Merkmale ausgewählt und gewichtet, um die Erkennungsrate zu maximieren und gleichzeitig die Fehlerrate zu minimieren.

\textbf{Kaskadenklassifikation}: Die Klassifikatoren werden in einer Kaskade organisiert, wobei jeder Klassifikator die Aufgabe hat, ein Fenster entweder als Gesicht oder Nicht-Gesicht zu klassifizieren. Ein Fenster, das von einem Klassifikator als Nicht-Gesicht klassifiziert wird, wird sofort verworfen, was die Berechnungen erheblich beschleunigt. Nur Fenster, die von allen Klassifikatoren in der Kaskade als Gesicht erkannt werden, werden letztendlich als Gesicht klassifiziert.

Die ersten Versuche konzentrierten sich darauf, Gesichter in verschiedenen Beleuchtungssituationen und Winkeln zu erkennen, um die Robustheit des Ansatzes zu testen.


\subsection{Vorteile und Nachteile}
Nach der Implementierung des Haar-Cascade-Ansatzes in OpenCV wurde die Methode intensiv getestet, um ihre Vor- und Nachteile zu ermitteln:

\textbf{Vorteile}:
\begin{itemize}
    \item \textbf{Schnelle Berechnung}: Die Methode ist sehr effizient in der Berechnung und kann in Echtzeit auf Geräten mit begrenzten Ressourcen wie dem Raspberry Pi ausgeführt werden.
    \item \textbf{Einfache Implementierung}: OpenCV bietet vorgefertigte Haar-Cascade-Modelle, die leicht zu integrieren sind.
\end{itemize}

\textbf{Nachteile}:
\begin{itemize}
    \item \textbf{Begrenzte Genauigkeit}: Die Genauigkeit der Erkennung ist begrenzt, insbesondere bei schwierigen Lichtverhältnissen oder seitlich aufgenommenen Gesichtern.
\end{itemize}

\subsection{Ergebnisse der Tests}

 Der Problem lag in der mangelhaften Genauigkeit, insbesondere bei schwierigen Lichtverhältnissen oder seitlich aufgenommenen Gesichtern. Diese Einschränkung lässt sich dadurch erklären, dass Haar-Cascades stark auf Kontraste und einfache geometrische Merkmale angewiesen sind. Bei wechselnden Lichtverhältnissen ändern sich die Intensitätsunterschiede im Bild, was dazu führt, dass die Merkmale, die für die Erkennung verwendet werden, weniger zuverlässig sind. Dies beeinträchtigt die Genauigkeit der Erkennung erheblich, da die Algorithmen Schwierigkeiten haben, die relevanten Merkmale konsistent zu identifizieren.

Zusammenfassend zeigte sich, dass die Haar-Cascade-Methode zwar effizient in der Berechnung ist und schnell auf Geräten mit begrenzten Ressourcen wie dem Raspberry Pi ausgeführt werden kann, jedoch nicht die erforderliche Präzision und Robustheit für die Gesichtserkennung in einem Smart Mirror bietet. Diese Erkenntnisse führten zur Entscheidung, nach präziseren Methoden für die Gesichtserkennung zu suchen.




\section{Umstieg auf Dlib für höhere Präzision}

\subsection{Wechsel zu Dlib}
Nach den begrenzten Erfolgen mit Haar-Cascades entschied man sich auf die Dlib-Bibliothek umzusteigen, um eine genauere Gesichtserkennung zu erreichen. Dlib ist eine freie Software-Bibliothek, die Algorithmen für maschinelles Lernen, Bildverarbeitung und maschinelles Sehen bereitstellt. Für die Gesichtserkennung in diesem Projekt wurde insbesondere der Histogram of Oriented Gradients (HOG)-Algorithmus und das 68-Facial-Landmarks-Modell zur präzisen Merkmalsextraktion verwendet.

\subsection{Technologien: HOG und 68-Facial-Landmarks}
\textbf{Histogram of Oriented Gradients (HOG)}: Der HOG-Algorithmus ist eine Methode zur Merkmalserkennung in Bildern, die darauf basiert, das lokale Auftreten von Gradientenorientierungen zu zählen. Diese Methode funktioniert folgendermaßen:

\begin{itemize}
    \item \textbf{Gradientenberechnung}: Für jedes Pixel im Bild wird der Gradient berechnet. Der Gradient eines Pixels gibt die Richtung und die Stärke der größten Helligkeitsänderung an\cite{hog_quelle}. Dies geschieht durch die Anwendung von Sobel-Filtern in horizontaler und vertikaler Richtung, wodurch zwei Bilder entstehen, die die Helligkeitsänderungen in x- und y-Richtung darstellen. Der Gradient kann dann durch die Kombination dieser beiden Bilder berechnet werden.
    
    \item \textbf{Zellaufteilung}: Das Bild wird in kleine Zellen unterteilt, typischerweise von 8x8 Pixeln. Diese Zellen sind klein genug, um lokale Details zu erfassen, aber groß genug, um signifikante Informationen zu enthalten. Innerhalb jeder Zelle werden die Gradientenorientierungen der Pixel gesammelt und analysiert.
    
    \item \textbf{Orientierungshistogramme}: Für jede Zelle wird ein Histogramm der Gradientenorientierungen erstellt. Die Gradienten innerhalb jeder Zelle werden in Bins sortiert, die verschiedene Richtungen repräsentieren, üblicherweise in 9 Bins, die Winkelbereiche von 0 bis 180 Grad abdecken. Jeder Bin enthält die Summe der Gradientenstärken, die in seine Richtung fallen, was eine robuste Darstellung der Orientierungsmuster innerhalb der Zelle ermöglicht.
    
    \item \textbf{Normierung}: Um die Beleuchtungsunterschiede zu kompensieren, werden die Histogramme in Blöcken normalisiert. Ein Block besteht aus mehreren benachbarten Zellen, typischerweise 2x2. Die Normierung erfolgt durch Berechnung der Quadratwurzel der Summe der quadrierten Bin-Werte, was eine gleichmäßige Darstellung der Merkmale ermöglicht, unabhängig von lokalen Beleuchtungsunterschieden.
    
    \item \textbf{Merkmalsvektor}: Die normalisierten Histogramme aller Blöcke werden zu einem einzigen Merkmalsvektor zusammengefügt, der das Bild repräsentiert. Dieser Merkmalsvektor ist hochdimensional und enthält eine detaillierte Beschreibung der Gradientenorientierungen im gesamten Bild. Er dient als Eingabe für maschinelle Lernalgorithmen, die darauf trainiert sind, Gesichter von Nicht-Gesichtern zu unterscheiden.
\end{itemize}

\begin{figure}[h!]
    \centering
    \includegraphics[width=0.5\textwidth]{pictures/hog.jpg}
    \caption{Beispiel einer HOG-Analyse}
    \label{fig:hoganalysis}
    \cite{hoganalysis}
\end{figure}

\textbf{68-Facial-Landmarks}:Da die fehlerfreie Erkennung von verschiedenen Gesichtern ein wichtiges Ziel war, wurde  zusätzlich zur Gesichtserkennung mit HOG das 68-Facial-Landmarks-Modell von Dlib verwendet, um präzise Merkmale innerhalb des erkannten Gesichts zu extrahieren. Diese Landmarks umfassen wichtige Gesichtspunkte wie Augen, Augenbrauen, Nase, Mund und Kieferlinie.

\subsection{Implementierung und Herausforderungen}
Die Implementierung von Dlib für die Gesichtserkennung begann mit der Integration der Bibliothek und dem Einbinden der vortrainierten Modelle für HOG und Facial Landmarks. Dies stellte sich jedoch als Herausforderung heraus, da Dlib zwei große Dateien (68-landmarks-modell: ~100MB; resnet-modell: ~21MB) benötigt, die die Modelle zur Gesichtserkennung und Merkmalsextraktion enthalten. Auf dem Raspberry Pi führte dies aufgrund des begrenzten RAMs von einem GB zu erheblichen Leistungsproblemen.

\subsection{Verbesserung der Performance}
Die initiale Implementierung mit Dlib ergab in qualifizierten Tests eine Performance von maximal einer Ausführung pro Sekunde. Um die Performance zu verbessern, wurden verschiedene Techniken implementiert:

\begin{itemize}
    \item \textbf{Reduzierung der Bildgröße}: Durch die Reduzierung der Bildgröße vor der Verarbeitung soll die Berechnungszeit verringert werden.
    \item \textbf{Frame Skipping}: Nicht jeder Frame wurde überprüft, um die Verarbeitungslast zu reduzieren.
    \item \textbf{Multithreading}: Implementierung von Multithreading zur gleichzeitigen Verarbeitung mehrerer Aufgaben.
\end{itemize}

Trotz Implementierung all dieser Techniken konnte die Performance auf nur maximal zwei Ausführungen pro Sekunde angehoben werden. Die Lösung kam letztendlich durch einen Rat von Professor Metzner: "[...]Es ist egal, dass es so langsam läuft, eine Gesichtsabfrage 1 mal pro Sekunde ist völlig ausreichend.[...]" Diese pragmatische Einstellung ermöglichte es mir, mich auf die Genauigkeit und Robustheit der Gesichtserkennung zu konzentrieren, anstatt auf die Geschwindigkeit.


\section{Feature Extraction und Matching}

\subsection{Feature Extraction}
Um zu überprüfen, ob ein Gesicht neu oder bereits im System bekannt ist, wurde Dlib's Deep Metric Learning Ansatz verwendet. Dieser Ansatz dient der Feature Extraction und nutzt das ResNet-Modell von Dlib, das speziell für die Gesichtserkennung trainiert wurde.

\textbf{Vorgehensweise}:
\begin{itemize}
    \item \textbf{Landmark-Detektion}: Nach der Gesichtserkennung werden die Gesichtspunkte mittels des vorher bereits erwähnten Modells (shape\_predictor\_68\_face\_landmarks.dat) bestimmt. Dieses Modell erkennt 68 charakteristische Punkte im Gesicht, wie Augen, Nase, Mund und Kieferlinie. Diese Punkte werden als Landmarks bezeichnet und helfen dabei, die Position und Ausrichtung des Gesichts zu bestimmen und dienen als Eingabe in das ResNet-Modells.
    
    \item \textbf{Berechnung des Gesichtsembeddings}: Mit Hilfe des Dlib-ResNet-Modells (dlib\_face\_recognition\_resnet\_model\_v1.dat) wird aus den extrahierten Landmarken ein Gesichtsembedding berechnet. Das ResNet-Modell verwendet die Informationen der 68 Landmarks, um einen 128-dimensionalen Vektor zu erzeugen. Dieser Vektor, das sogenannte Gesichtsembedding, repräsentiert die einzigartigen Merkmale eines Gesichts in einem hochdimensionalen Raum. Die Werte des Vektors sind als Floating Points zwischen 0 und 1 skaliert.
\end{itemize}

\begin{figure}[h!]
    \centering
    \includegraphics[width=0.5\textwidth]{pictures/68landmarks.jpg}
    \caption{Praxisbeispiel der 68-landmarks}
    \label{fig:68landmarks}
    \cite{68landmarks}
\end{figure}

\subsection{Vergleich der Gesichtsembeddings}
Die erstellten Embeddings werden zur Identifikation von Personen genutzt, indem sie mit bereits gespeicherten Embeddings verglichen werden. 

\textbf{Vergleichsmethode}:
Zur Identifikation wird der euklidische Abstand zwischen den Embeddings berechnet. Der euklidische Abstand ist eine Maßzahl für die Distanz zwischen zwei Punkten in einem n-dimensionalen Raum, die durch die Wurzel der Summe der quadrierten Differenzen ihrer Koordinaten berechnet wird. Wenn der Abstand gering ist (unter einem bestimmten Schwellenwert, z.B. 0.6), wird angenommen, dass es sich um dieselbe Person handelt. Der Schwellenwert von 0,6 wurde durch eine Reihe von Tests empirisch bestimmt, um eine Balance zwischen False Positives und False Negatives zu erreichen, was eine zuverlässige Unterscheidung zwischen verschiedenen Personen ermöglichen soll.

\section{Speichern und Verwalten der Profile}

An diesem Zeitpunkt im Projekt war der Kern der Gesichtserkennung, also das Erkennen eines Gesichts, das Erstellen eines Face Embeddings und die Logik für den Vergleich mit bereits erkannten Gesichtern fertiggestellt. Es folgte also nun die Integration in unser bestehendes System. Die Kommunikation mit dem Display-Webserver lief reibungslos, da dieser auf dem gleichen Gerät gehostet wird und somit nur der Ablagepfad der Profildaten geändert werden musste. Ein größeres Problem war jedoch die Kommunikation mit der Remote-App. Da diese ebenfalls Änderungen an den Profilen vornehmen kann, mussten wir sicherstellen, dass die Profile samt Einstellungen auf beiden Seiten synchron sind, um Informationsverlust oder Schlimmeres zu verhindern. Dafür gab es einige Besprechungen mit dem Entwickler der App, David Vollmer. Wir einigten uns auf ein genormtes Format der Profildatenspeicherung (profiles.json), welches dann über einen Websocket vom Spiegel an die Remote und vice versa gesendet werden kann.
\subsection{Speichern der Profile}
 Wenn ein unbekanntes Gesichtsembedding identifiziert wird, wird ein neues Profil mit Standardwerten in der JSON-Datei erstellt. Diese Datei enthält die Informationen darüber, welche Profile existieren, welches Profil gerade aktiv ist und für jedes Profil, welche Widgets (IDs) an welcher Stelle (index) angezeigt werden sollen. Wird ein Gesichtsembedding wiedererkannt, wird jediglich das dazugehörige 'isSelected' Flag gesetzt und das vorherige gecleared.


\subsection{Verbindung zum Websocket}
Um die Profile mit der Android-Remote zu synchronisieren wurde eine Websocket-Verbindung implementiert. Diese Verbindung ermöglicht die Echtzeitsynchronisation von Profiländerungen zwischen dem Smart Mirror und der Android-App.

\textbf{Vorgehensweise}:
\begin{itemize}
    \item \textbf{Websocket-Listener}: Sowohl die Remote-App als auch der Smart Mirror haben jeweils einen Listener am Websocket, um Nachrichten zu empfangen.
    
    \item \textbf{Synchronisation beim App-Start}: Beim Starten der Android-App sendet sie einen 'fetch'-Befehl über den Websocket, um sicherzustellen, dass sie sofort mit den aktuellen Profildaten synchronisiert wird. Der Smart Mirror antwortet daraufhin mit der aktuellen profiles.json-Datei.

    \item \textbf{Automatische Updates bei Profiländerungen}: Die profiles.json wird jedes Mal von dem Spiegel an den Websocket gesendet, wenn:
    \begin{itemize}
        \item Ein neues Profil angelegt wurde (das neue Profil erhält automatisch das 'isSelected'-Flag).
        \item Ein neues, aber bereits bekanntes Gesicht erkannt wird und das 'isSelected'-Flag entsprechend gesetzt wurde.
    \end{itemize}

    \item \textbf{Updates von der Remote-App}: Wenn die Remote-App eine Änderung an einem Profil vornimmt, sendet sie die profiles.json an den Websocket. Der Smart Mirror empfängt diese Nachricht und ersetzt die bestehende profiles.json mit den neuen Daten.
\end{itemize}

\section{Schlussfolgerung}



\subsection{Ausblick}
Die Entwicklung der Gesichtserkennungslösung hat eine solide Grundlage geschaffen, auf der zukünftige Erweiterungen und Verbesserungen aufbauen können. Obwohl einige vielversprechende Ideen aufgrund der Stabilität des Systems nicht umgesetzt wurden, bieten sie spannende Möglichkeiten für zukünftige Arbeiten.

\begin{itemize}
    \item \textbf{Age- und Gender-Classifier}: Die Dlib-Bibliothek bietet auch Age- und Gender-Classifier, die in Zusammenarbeit mit dem Widget-Team als eigenständiges Widget hätten angeboten werden können. Lokale Tests hatten dazu bereits funktioniert, und eine vollständige Implementierung könnte die Personalisierung des Smart Mirrors weiter verbessern.
    
    \item \textbf{Änderung des Profilnamens}: Mit mehr Zeit wäre eine Möglichkeit implementiert worden, den Profilnamen zu ändern. Derzeit ist der Profilname der einzigartige Bezeichner für die Face Embeddings. Um dies zu ermöglichen, müsste ein zusätzlicher eindeutiger Bezeichner eingeführt und die Profil-Logik sowohl auf dem Spiegel als auch in der App angepasst werden.
    
    \item \textbf{Flusskontrolle}: Eine weitere wichtige Verbesserung wäre die Einführung einer Flusskontrolle. Derzeit senden Spiegel und App Änderungen sofort, wenn bestimmte Ereignisse auftreten. Dies kann zu Konflikten und inkonsistenten Daten führen, wenn beide Seiten gleichzeitig senden. Eine Flusskontrolle könnte sicherstellen, dass nur eine Seite senden kann, während die andere Seite blockiert ist. Gesendete Änderungen während der Blockierphase würden in einem Puffer gespeichert und nach der Blockierung sofort gesendet.
    
    \item \textbf{Optimierung des Schwellenwerts}: Mit mehr Zeit hätten intensivere Tests zur Einstellung des euklidischen Abstands durchgeführt werden können, um Fehler wie bei der Produktpräsentation zu vermeiden. Ein genauer kalibrierter Schwellenwert könnte die Erkennungsgenauigkeit weiter verbessern und sicherstellen, dass neue Gesichter nicht fälschlicherweise als bekannte Profile erkannt werden.
    
    \item \textbf{Datenbanksynchronisation}: Wenn wir vorher gewusst hätten, wie aufwändig die Synchronisation der Profile sein würde, hätten wir definitiv eine Datenbank verwendet , die die Synchronisation eigenständig durchführt. Eine Datenbank könnte viele der aktuellen Herausforderungen bei der Datenkonsistenz und -synchronisation lösen.
\end{itemize}

\subsection{Zusammenfassung}
Im Verlauf dieses Projekts wurde eine fortschrittliche Gesichtserkennungslösung für einen Smart Mirror entwickelt. Der Übergang von herkömmlichen Haar-Cascades zu modernen Methoden wie Dlib und HOG führte zu einer signifikanten Verbesserung der Erkennungsgenauigkeit. Trotz der Performance-Probleme konnte eine zufriedenstellende Verarbeitungsgeschwindigkeit erreicht werden.

Ein wichtiger Teil des Projekts war die Speicherung und Verwaltung der Profildaten sowie die Synchronisation mit einer Android-Remote-App über eine Websocket-Verbindung. Die Trennung der Face Embeddings von der profiles.json-Datei und die Implementierung eines robusten Synchronisationsmechanismus stellten sicher, dass die Daten auf beiden Seiten konsistent und aktuell blieben.

Insgesamt konnte eine zuverlässige Gesichtserkennung und Profilverwaltung implementiert werden.


\chapter{Schnittstelle}

In diesem Kapitel wird die Schnittstelle des \textbf{Spiegel AI} Projekts beschrieben. Wir erläutern die verschiedenen Schnittstellen, die verwendet werden, sowie deren Implementierung und Nutzung.

\section{Überblick}
Geben Sie einen Überblick über die verwendeten Schnittstellen.

\section{Implementierung}
Beschreiben Sie die Implementierung der Schnittstellen.

\section{Nutzung}
Erläutern Sie, wie die Schnittstellen genutzt werden.


\chapter{Ergebnisse}

In diesem Kapitel fassen wir die Ergebnisse des Projekts \textbf{Spiegel AI} zusammen. Wir gehen auf die erreichten Ziele, die Herausforderungen und die zukünftigen Arbeiten ein.

\section{Zusammenfassung der Ergebnisse}

\subsection{Erreichte Ziele}
Unser Projektteam hat den Smart Mirror erfolgreich realisiert und dabei folgende Ziele erreicht:

Der Spiegelprototyp wurde ansehnlich und robust aufgebaut. Er ist nicht nur funktional, sondern fügt sich auch optisch ansprechend in die Umgebung ein. Die Display-Ausgabe funktioniert einwandfrei und ermöglicht eine klare und deutliche Darstellung aller Inhalte.

Der Smart Mirror ist in der Lage, die aktuelle Wetterlage des Ortes sowie Kalendereinträge anzuzeigen und bietet viele weitere informative und nützliche Funktionen. Benutzer des Spiegels haben die Möglichkeit, ihr angezeigtes Profil individuell zu gestalten. Dies erfolgt durch die Auswahl aus neun verschiedenen Widgets, die auf acht unterschiedlichen Positionen platziert werden können. Diese Personalisierung ermöglicht eine hohe Flexibilität und Anpassungsfähigkeit an die individuellen Bedürfnisse und Vorlieben der Benutzer.

Ein weiteres herausragendes Merkmal ist die interaktive Steuerung über eine mobile Kontroll-App. Diese App ermöglicht es den Benutzern, ihr Profil zu ändern oder zu personalisieren, was die Benutzerfreundlichkeit und den Komfort erheblich steigert. Alle Informationen werden in deutscher Sprache angezeigt, was die Bedienung für deutschsprachige Nutzer intuitiv und einfach macht.

Ein neues Gesicht wird erkannt und das jeweilige Profil angezeigt, was eine schnelle und reibungslose Nutzung sicherstellt. Die Benutzeroberfläche wurde in Tests als 'sehr intuitiv' eingestuft, was die hohe Benutzerfreundlichkeit des Systems unterstreicht.

Technisch konnten wir sicherstellen, dass sowohl der Spiegel als auch die App erfolgreich eine Verbindung zum Websocket herstellen können. Beide Komponenten senden nach einer ausgeklügelten Synchronisationslogik jeweils den aktuellen Stand der Profile, wodurch eine konsistente und aktuelle Datenbasis gewährleistet wird. Die Gesichtserkennung arbeitet äußerst zuverlässig und erkennt Personen zu 95 Prozent korrekt, wodurch die personalisierten Profile geladen und angezeigt werden können.

\subsection{Herausforderungen}
Die Herausforderungen, denen sich das Team während der Entwicklung des Projekts stellen musste, sind in den jeweiligen Dokumentationen der einzelnen Projektteilnehmer ausführlich beschrieben. 

\subsection{Zukünftige Arbeiten}
Für zukünftige Arbeiten und mögliche Erweiterungen des Projekts verweisen wir ebenfalls auf die detaillierten Ausführungen in den Dokumentationen der einzelnen Projektteilnehmer.

\subsection{Betriebssetzung}
Zum Verwenden des Spiegel AIs sind einige Schritte notwendig. Zuerst benötigen Sie eine Kamera und einen WLAN-Stick, da diese Komponenten Eigentum der Projektteilnehmer waren, welche sie wieder mitnahmen. Beim Verwenden eines WLAN-Sticks ist noch zu beachten, dass gegebenenfalls Treiber installiert werden müssen. Nachdem die Hardware bereit ist, sind die folgenden Schritte zur Betriebssetzung des Smart Mirrors nötig:
\begin{enumerate}
	\item Schließen Sie alle über USB zu verbindenden Komponenten, inklusive Tastatur und Maus, am Raspberry Pi an.
	\item Schalten Sie den Raspberry Pi ein.
	\item Öffnen Sie ein Terminal-Fenster und navigieren Sie zu \verb|~/Mirror/dt-g5/|.
	\item Öffnen Sie zwei neue Tabs.
	\item Navigieren Sie im ersten Tab zu \verb|~/Mirror/dt-g5/websocket_server| und starten Sie den Websocket mit \texttt{python websocket.py}.
	\item Navigieren Sie im zweiten Tab zu \verb|~/Mirror/dt-g5/Display| und starten Sie einen HTTP-Server mit \texttt{python -m http.server 8001}.
	\item Navigieren Sie im dritten Tab zu \\ \verb|~/Mirror/dt-g5/facialRec/facialRecReadability/mitWebsocket| und starten Sie die Gesichtserkennung mit \texttt{python main.py}
	\item Öffnen Sie den Chromium Browser und rufen Sie \texttt{http://localhost:8001} auf.
	\item Eventuell müssen Sie hier den Cache bereinigen, falls die Anzeige sich nicht ändert.
	\item Installieren Sie die Android-APK auf einem Smartphone. Sie ist im GitLab-Repository im Verzeichnis \texttt{remote\_app} hinterlegt.
	\item Starten Sie die Applikation auf dem Smartphone.
\end{enumerate}


\chapter*{Stundenliste}
\addcontentsline{toc}{chapter}{Stundenliste}

\begin{longtable}[c]{|c|>{\raggedright\arraybackslash}p{2.5cm}|>{\raggedright\arraybackslash}p{7cm}|}
\caption*{\textbf{Stundenliste Leon Kranner}} \\
\hline
\textbf{Kalenderwoche} & \textbf{Stunden} & \textbf{Aufgabe} \\
\hline
\endfirsthead

\multicolumn{3}{c}%
{\tablename\ \thetable{} -- Fortsetzung von vorheriger Seite} \\
\hline
\textbf{Kalenderwoche} & \textbf{Stunden} & \textbf{Aufgabe} \\
\hline
\endhead

\hline \multicolumn{3}{r}{{Fortsetzung auf nächster Seite}} \\
\endfoot

\hline
\endlastfoot

12 & 3 & Einführungfsveranstaltung \\
\hline
13 & 3 & GANNT Diagramm \\
   & 3 & Teambesprechung \\
\hline
14 & 2 & Planung mit Hardware-Team \\
   & 2 & Teambesprechung \\
\hline
15 & 3 & Postererstellung und Hw \\
   & 1 & Displaymessung + postervorstellung \\
   & 2 & Teambesprechung \\
\hline
16 & 4 & Display-Projekt aufsetzen \\
   & 2 & Baumarkt Materialien erkunden \\
\hline
17 & 2 & Umstrukturierung des Display-Projekts \\
   & 2 & Neues Widget erstellen \\
\hline
18 & 6 & Weitere Widgets und Änderungen an alten Widget \\
\hline
19 & 3 & Einkerbungen fräsen \\
   & 1 & MDF Platte auf Maß schneiden \\
\hline
20 & 2 & Teambesprechung \\
   & 1 & Umstrukturierung des Stundenplans \\
\hline
21 & 2 & Teambesprechung \\
\hline
22 & 2 & Teambesprechung \\
\hline
23 & 2 & Automatische Aktualisierung der Widgets \\
\hline
24 & 2 & Teambesprechung \\
   & 2 & Besprechung Schnittstellen \\
\hline
25 & 2 & Teambesprechung \\
   & 1 & Plexiglas überarbeiten \\
\hline
26 & 2 & Teambesprechung \\
   & 3 & Austausch der Spiegel Folie, Aufbau der Spiegels, Hardware installieren \\
   & 1 & Testen des Displays mit Aufgebauten Spiegels \\
\hline
27 & 2 & Powerpoint erstellung \\
   & 3 & HMTL neu anordnen auf Basis von Json Datei \\
   & 2 & Besprechung profiles sync \\
\end{longtable}

Gesamtstunden: 109

% Neue Seite für die nächste Tabelle
\newpage


\begin{longtable}[c]{|c|>{\raggedright\arraybackslash}p{2.5cm}|>{\raggedright\arraybackslash}p{7cm}|}
\caption*{\textbf{Stundenliste Marco Kuner}} \\
\hline
\textbf{Kalenderwoche} & \textbf{Stunden} & \textbf{Aufgabe} \\
\hline
\endfirsthead

\multicolumn{3}{c}%
{\tablename\ \thetable{} -- Fortsetzung von vorheriger Seite} \\
\hline
\textbf{Kalenderwoche} & \textbf{Stunden} & \textbf{Aufgabe} \\
\hline
\endhead

\hline \multicolumn{3}{r}{{Fortsetzung auf nächster Seite}} \\
\endfoot

\hline
\endlastfoot

12 & 3 & Einführungfsveranstaltung \\
\hline
13 & 3 & GANNT Diagramm \\
\hline
14 & 3 & Teileliste / GANNT Diagramm \\
   & 2 & Teambesprechung \\
   & 3 & Inventur \\
\hline
15 & 10 & Technologie-Recherche + Postererstellung \\
   & 3 & Posterdemütigung ertragen und HW \\
   & 2 & Teambesprechung \\
\hline
16 & 2 & Teambesprechung \\
   & 8 & Erster Prototyp mit HAAR Cascades \\
\hline
17 & 2 & Teambesprechung \\
   & 8 & Neue Version mit DLIB Bibs geschrieben \\
\hline
18 & 2 & Teambesprechung \\
   & 2 & Recherche über facial Landmark Storage \\
   & 4 & Neue Iteration mit Storage Technologie \\
\hline
19 & 10 & Troubleshoot da extrem langsam \\
   & 2 & Teambesprechung \\
\hline
20 & 2 & Teambesprechung \\
   & 4 & Recherche zu geeigneter Schnittstelle und Format der Profilerstellung mit profile landsmarks \\
\hline
21 & 2 & Teambesprechung \\
\hline
22 & 2 & Teambesprechung \\
\hline
23 & 2 & Teambesprechung \\
\hline
24 & 2 & Teambesprechung \\
\hline
25 & 2 & Teambesprechung \\
   & 6 & Implementieren einer Lösung zur automatischen Erkennung eines neuen Gesichts und output der Daten in .json \\
   & 2 & Schnittstellen Thinktank mit David \\
\hline
26 & 2 & Teambesprechung \\
   & 5 & Ausgabe und automatische Aktualisierung einer genormten profiles.json \\
   & 6 & Implementierung eines neuen Websockets zwischen Raspi und Android in Vorbereitung zur Synchronisation \\
   & 2 & Recherche zu Technologien zur Synchronisation zwischen Raspi und Android (inotify?) \\
   & 2 & Besprechung mit remote app Spezialist bzgl. Synchronisationsproblemen \\
\hline
27 & 4 & Vor- und Aufbereiten der Präsentation \\
   & 2 & Verbessern der readability des Algorithmus \\
   & 8 & Implementation des Websockets mitsamt Logik für andauernder Synchronisation \\
   & 4 & Troubleshooting: Gesichtserkennung stürzt ab auf Raspi \\
\end{longtable}

Gesamtstunden: 130

% Neue Seite für die nächste Tabelle
\newpage


\begin{longtable}[c]{|c|>{\raggedright\arraybackslash}p{2.5cm}|>{\raggedright\arraybackslash}p{7cm}|}
\caption*{\textbf{Stundenliste David Vollmer}} \\
\hline
\textbf{Kalenderwoche} & \textbf{Stunden} & \textbf{Aufgabe} \\
\hline
\endfirsthead

\multicolumn{3}{c}%
{\tablename\ \thetable{} -- Fortsetzung von vorheriger Seite} \\
\hline
\textbf{Kalenderwoche} & \textbf{Stunden} & \textbf{Aufgabe} \\
\hline
\endhead

\hline \multicolumn{3}{r}{{Fortsetzung auf nächster Seite}} \\
\endfoot

\hline
\endlastfoot

12 & 3 & Einführungsveranstaltung \\
   & 4 & Setup Gitlab und Drafts \\
\hline
13 & 4 & Erstellung GANNT Diagramm und Lastenheft \\
   & 2 & Teambesprechung \\
\hline
14 & 5 & Abgabevorbereitung GANNT und Lastenheft \\
   & 2 & Teambesprechung \\
   & 3 & Hardwarediskussion und -suche \\
   & 2 & Überarbeitung GANNT und Lastenheft \\
\hline
15 & 4 & Postererstellung \\
   & 3 & Vostellung Poster und Hardwaresuche \\
   & 6 & Setup Flutter und Frontend dev \\
   & 2 & Teambesprechung \\
   & 2 & Frontend dev (Navigation) \\
\hline
16 & 2 & Teambesprechung \\
   & 3 & Frontend dev \\
\hline
17 & 2 & Setup Raspberry Pi \\
   & 2 & Teambesprechung \\
\hline
18 & 2 & Teambesprechung \\
\hline
19 & 2 & Teambesprechung \\
   & 1 & Frontend dev (Widget buttons) \\
\hline
20 & 2 & Teambesprechung \\
\hline
21 & 2 & Teambesprechung \\
\hline
22 & 2 & Teambesprechung \\
   & 4 & Troubleshooting Android SDK \\
\hline
23 & 2 & Teambesprechung \\
\hline
24 & 2 & Besprechung Schnittstellen \\
   & 4 & Frontend dev (Widgets final) \\
   & 2 & Teambesprechung \\
   & 2 & Konfiguration Raspberry Pi \\
   & 8 & Konfiguration Schnittstellen (Flutter + Server) \\
   & 5 & Konfiguration Schnittstellen (Spiegel + Server) \\
\hline
25 & 2 & Teambesprechung \\
   & 3 & Konfiguration Raspberry Pi wifi \\
   & 5 & Troubleshooting + Testing Websocket \\
   & 2 & Anpassung Android und iOS (icon, splash, usw.) \\
   & 8 & Anpassung Datenspeicher, Profile und Websocket \\
   & 3 & Speichern von Widget- und Remotestatus in Profilen \\
   & 3 & Code Refactoring und Bugfixing \\
\hline
26 & 1 & Besprechung Schnittstellen Profile \\
   & 1 & Überarbeitung Poster \\
   & 2 & Teambesprechung \\
   & 1 & Überarbeitung Websocket-Message \\
   & 2 & Troubleshooting selected Widgets \\
   & 2 & Konfiguration Raspberry Pi Gesichtserkennung \\
   & 6 & Erstellung Powerpoint \\
\hline
27 & 2 & Besprechung profiles sync \\
   & 1 & Refactoring File Reader \\
   & 2 & Implementierung profiles sync \\
   & 3 & Schreiben des Präsentationsskripts \\
   & 2 & Übung Präsentation \\
   & 5 & Testen der Gesichtserkennung am Websocket \\
   & 3 & Testen und Korrigieren profiles sync \\
\end{longtable}

Gesamtstunden: 150

% Neue Seite für die nächste Tabelle
\newpage

\begin{longtable}[c]{|c|>{\raggedright\arraybackslash}p{2.5cm}|>{\raggedright\arraybackslash}p{7cm}|}
\caption*{\textbf{Stundenliste Marcel Wagner}} \\
\hline
\textbf{Kalenderwoche} & \textbf{Stunden} & \textbf{Aufgabe} \\
\hline
\endfirsthead

\multicolumn{3}{c}%
{\tablename\ \thetable{} -- Fortsetzung von vorheriger Seite} \\
\hline
\textbf{Kalenderwoche} & \textbf{Stunden} & \textbf{Aufgabe} \\
\hline
\endhead

\hline \multicolumn{3}{r}{{Fortsetzung auf nächster Seite}} \\
\endfoot

\hline
\endlastfoot

12 & 3 & Einführungfsveranstaltung \\
\hline
13 & 3 & GANNT Diagramm \\
   & 2 & Teambesprechung \\
   & 2 & Vorbereitung Template \\
\hline
14 & 2 & Teambesprechung \\
   & 3 & Planung und Hardware Suche \\
   & 3 & Setup CAD und ersten Entwurf zeichnen \\
\hline
15 & 3 & Poster Erstellung \\
   & 2 & Besprechung Spiegelrahmen \\
   & 2 & Detaillierung der CAD Datei \\
   & 1 & Baumarkt \\
   & 2 & Teambesprechung \\
\hline
16 & 3 & Einführung Display Programmierung und erste Ansätze \\
   & 2 & Planung und Besprechung für den Bilderrahmen \\
   & 2 & Materialien im Baumarkt suchen \\
   & 2 & Teambesprechung \\
\hline
17 & 4 & Weitere Setup für Display Programmierung \\
   & 4 & Projekt Besprechung und weitere Programmierung \\
   & 4 & Weitere Widget Programmierung und Bug Fixing \\
   & 2 & Teambesprechung \\
\hline
18 & 2 & Display Programmierung (Fertigstellung des Verkehrsinformations Widget) \\
   & 2 & Teambesprechung \\
   & 4 & Holz auf Maß schneiden und hobeln \\
   & 3 & Vorbohren und Bilderrahmen zurechtlegen \\
\hline
19 & 3 & Einkerbungen fräsen \\
   & 1 & MDF Platte auf Maß schneiden \\
   & 1 & In MDF Platte Löcher bohren für Befestigung \\
   & 1 & Zwischenstücke vorne anschrauben \\
   & 2 & Teambespechung \\
   & 2 & Holz zusammenleimen und trocknen lassen \\
\hline
20 & 2 & Teambesprechung \\
   & 1 & Bug Fixing von älteren Widgets \\
\hline
21 & 2 & Teambesprechung \\
   & 1 & Plexiglas auf Maß schneiden \\
\hline
22 & 2 & Teambesprechung \\
\hline
23 & 3 & Erstellung weiter Widgets \\
   & 2 & Teambespechung \\
\hline
24 & 2 & Teambesprechung \\
   & 2 & Besprechung Schnittstellen \\
   & 2 & Bugfixing für das News Widget \\
\hline
25 & 2 & Teambesprechung \\
   & 1 & Bugfixing der Widget Ansicht \\
   & 1 & Plexiglas überarbeiten \\
   & 2 & Löcher bohren und Plexiglas festschrauben \\
   & 1 & Spiegelfolie aufbringen und Kabel Loch bohren \\
   & 5 & Schnittstelle zwischen Gesichtserkennung und Display die Grundlagen auf Seite des Displays aufsetzen \\
\hline
26 & 2 & Teambesprechung \\
   & 1 & Fehler korrigieren am Spiegel \\
   & 1 & Bugfixing Widget \\
   & 3 & Austausch der Spiegel Folie, Aufbau der Spiegels, Hardware installieren \\
   & 1 & Testen des Displays mit Aufgebauten Spiegels \\
   & 1 & Vorbereitung Präsentation \\
   & 2 & Vorbereitung Präsentation (Bilder und Aufbau Finalisieren) \\
   & 2 & Raspberry Pi Gesichtserkennung Testen \\
   & 5 & Profile aus der Gesichtserkennung auslesen und Speicherin in Raspberry \\
   & 2 & Bugfixing Browser Problem \\
\hline
27 & 2 & Powerpoint erstellung \\
   & 3 & HMTL neu anordnen auf Basis von Json Datei\\
   & 2 & Troubleshooting Chache Probleme \\
   & 1 & nicht ausgewählte Widgets ausblenden \\
   & 2 & Vorbereitung Präsentation \\
   & 3 & Schreiben des Präsentationsskripts \\
   & 2 & Teambespechung \\
   & 1 &  Studenliste in Dokumentation eintragen \\
   & 4 & Dokumentation für Widgets schreiben \\
   & 1 & Dokumentation für Testverfahren schreiben \\
   & 3 & Verschieden Testvarianten bei den Widgets durchführen \\
   & 2 & Dokumentation der Dynamischen Widget Anordnung / Überarbeitung Layout \\

\end{longtable}

Gesamtstunden: 151

\newpage

\chapter*{Literaturverzeichnis}

\printbibliography[heading=none]

\listoffigures

\end{document}
