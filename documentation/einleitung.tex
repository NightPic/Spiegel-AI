\chapter*{Einleitung}
\addcontentsline{toc}{chapter}{Einleitung}

In der Einleitung stellen wir das Projekt \textbf{Spiegel AI} vor. Wir beschreiben die Zielsetzung des Projekts, die Motivation und den allgemeinen Aufbau der Dokumentation. Zudem geben wir einen Überblick über die eingesetzte Hardware und Software sowie die geplanten Anwendungsbereiche.

\section*{Zielsetzung}
Unser Ziel ist es, einen intelligenten Spiegel zu entwickeln, der durch seine benutzerfreundliche und interaktive Oberfläche den Alltag der Nutzer erleichtert. Er soll personalisierte Informationen bereitstellen und somit erkennen, welche Personen vor dem Spiegel sind. Der Nutzer soll durch die zugehörige App das Layout und die Widgets nach Belieben verändern können. Zudem soll der Spiegel automatisch neue Personen erkennen und ein neues Profil erstellen.

\section*{Motivation}
Die Motivation hinter der Entwicklung eines SMART Mirrors liegt in der Verbesserung des täglichen Lebens durch Effizienzsteigerung und Zeiteinsparung. Nutzer des intelligenten Spiegel sollen am Spiegel bereits die wichtigsten Informationen für den Tag bereitgestellt bekommen. So können Nutzer z.B. während ihrer Morgenroutine gleichzeitig die wichtigsten Informationen für den Tag ablesen.

\section*{Überblick}
Im ersten Kapitel gehen wir zunächst auf den Aufbau des Rahmens ein. Danach werden wir dieHardware des Projektes genauer ansehen. Dabei werden wir auf die Komponenten und Installation der Hardware den Fokus legen. Im 3. Abschnitt werden wir auf das Display und ihre Widgets eingehen. Des weiteren werden wir die Spiegel AI Remote App beleuchten. Dabei werden wir auf die Funktionen, Implementierung und die Testmöglichkeiten eingehen. Die dazugehörige Gesichtserkennung wird im 5. Kapitel beschrieben. Im nächsten Abschnitt werden wir die Schnittstelle zwischen App, Raspberry und Gesichtserkennung erläutern. Im letzten Abschnitt werden wir auf unsere Ergebnisse eingehen und ein Fazit daraus schließen. 
