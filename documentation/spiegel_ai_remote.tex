\chapter{Spiegel AI Remote}
\textit{Erarbeitet von David Vollmer.} \\ \\
Im folgenden wird die \textbf{Spiegel AI Remote} App - auch \textbf{Remote App} genannt - beschrieben. Es handelt sich dabei um eine mobile Anwendung, dessen Hauptaufgabe die Fernsteuerung des Smart Mirrors ist.

\section{Die Flutter\texttrademark{} SDK}
Für die Entwicklung einer mobilen Applikation gibt es heutzutage viele Tool-Kits, die verwendet werden können. Laut einer von JetBrains durchgeführten Umfrage war Flutter im Jahr 2023 das am häufigsten verwendete mobile, plattformübergreifende Framework.
\begin{figure}[h]
    \centering
    \includegraphics[width=0.4\textwidth]{pictures/frameworks_stats.png}
    \captionsetup{justification=centering, labelformat=simple, singlelinecheck=false}
    \caption{JetBrains Entwicklerumfrage 2023 \cite{jetbrains_survey}}
    \label{fig:jetbrains_survey}
\end{figure} \\
Zusätzlich zu ihrer Popularität ist die von Google entwickelte Flutter SDK in der Lage, mittels AOT-Compiler Programme direkt für die Zielplattform zu kompilieren. Dabei wird die Programmiersprache Dart verwendet.\cite{dart_platform} Flutter unterstützt unter anderem die Entwicklung auf den Plattformen Android SDK, iOS, Windows, macOS und Web.\cite{flutter_supported_platforms} Für die Umsetzung der Fernsteuerungs-App wurde insbesondere mit den Plattformen Android und iOS entwickelt und getestet.

\section{Funktionen}
Um das Display des Spiegel AIs fernzusteuern, müssen einige Hauptfunktionalitäten vorhanden sein. Die Remote App muss in der Lage sein, mit dem Spiegel zu kommunizieren, die Anzeige der Widgets auf dem Display zu ändern, verfügbare Widgets auszuwählen und Profile zu verwalten. Mit Ausnahme der ersten Anforderung, welche in der \textbf{Implementierung} und im Kapitel \textbf{Schnittstelle} näher beschrieben wird, werden all diese Punkte in sogenannten Ansichten (englisch: views) behandelt. Diese kann der Nutzer mithilfe einer Navigationsleiste auswählen.
\begin{figure}[h]
    \centering
    \begin{minipage}[b]{0.27\textwidth}
        \centering
        \includegraphics[width=\textwidth]{pictures/remote_remote.jpg}
        \captionsetup{justification=centering, labelformat=simple, singlelinecheck=false}
        \caption{Remote Ansicht}
    \end{minipage}
    \hfill
    \begin{minipage}[b]{0.27\textwidth}
        \centering
        \includegraphics[width=\textwidth]{pictures/remote_widgets.jpg}
        \captionsetup{justification=centering, labelformat=simple, singlelinecheck=false}
        \caption{Widgets Ansicht}
    \end{minipage}
    \hfill
    \begin{minipage}[b]{0.27\textwidth}
        \centering
        \includegraphics[width=\textwidth]{pictures/remote_profile.jpg}
        \captionsetup{justification=centering, labelformat=simple, singlelinecheck=false}
        \caption{Profile Ansicht}
    \end{minipage}
\end{figure}

\subsection{Remote View}
Die Remote View, welche die Standardansicht nach Öffnen der App ist, bietet die Möglichkeit, den Status der Displayanzeige am Spiegel zu ändern. In der Mitte wird der Name des gerade ausgewählten Profils angezeigt. Dieses Feld lässt keinerlei Interaktion zu, da das zentrale Feld der Spiegelanzeige frei bleibt. Das bedeutet, dass bis zu acht Widgets angezeigt und geändert werden können. Mit einem Klick auf einen Button wird das jeweilige Widget aus- oder eingeblendet. Ein Feld in grauer Farbe bedeutet, dass das Widget vom Spiegel AI Display nicht angezeigt wird. Zieht man ein Widget über ein anderes, werden ihre Positionen getauscht. Die Felder, die keinen Text enthalten, haben die selben Interaktionsmöglichkeiten wie die anderen. Sie sind Platzhalter für Widgets, die hinzugefügt werden können. Falls kein Profil ausgewählt ist, wird im Remote View eine Standardeinstellung angezeigt und jegliche Interaktion der Buttons ist ausgestellt. Bei einem Versuch, ohne Profilselektion eine Änderung vorzunehmen, wird eine Snackbar angezeigt, welche darauf verweist, dass ein Profil geladen sein muss.

\subsection{Widgets View}
In der Widgets Ansicht können für das ausgewählte Profil Widgets ausgewählt werden. Diese View bietet die Widgets Kalender, Uhr, Wetter, Notizen, Termine, Verkehr, Nachrichten, Tanken, Profil und TestWidget an. Bei letzterem handelt es sich um einen Platzhalter, welcher zum Testen der Widgetfunktionalitäten verwendet wurde, aber auch zukünftig mit einem neuen Widget ersetzt werden kann. Die Anwendungen der restlichen Widgets sind im Kapitel \textbf{Display} beschrieben. Mithilfe eines Toggle-Buttons werden bis zu acht Widgets selektiert. Beim Versuch, ein neuntes Widget auszuwählen, schlägt dies fehl und eine Snackbar benachrichtigt über die Obergrenze erlaubter Widgets. Auf die Änderung eines Widgets, ohne ein Profil geladen zu haben, folgt ebenfalls eine dementsprechende Fehlermeldung. Wird ansonsten ein Widget ausgeschaltet, dann wird das im Toggle-Button signalisiert und in der Remote View wird der Name des Widgets mit einem leeren Feld ersetzt. Wenn ein ausgeschaltetes Widget ausgewählt wird, aktualisiert sich auch da der Toggle-Schalter und in der Remote Ansicht wird das erste Feld ohne Textinhalt mit dem Namen des Widgets versehen.

\subsection{Profile View}
Die letzte navigierbare Ansicht ist die Profile View. Hier findet die Verwaltung der gespeicherten Profile statt. Die Profile werden aufgelistet und können mit einem Klick ausgewählt werden. Hält man ein Profil für eine kurze Zeit gedrückt, kann man diese in ihrer Position in der Auflistung ändern, indem man sie an die gewünschte Stelle zieht. Löschen kann man einen Eintrag, indem auf das Mülleimer-Icon geklickt wird. Darauf öffnet sich ein sogenanntes Alert-Dialog, welches das Abbrechen oder Bestätigen der Löschung durchführt. Ein neues Profil kann erstellt werden, indem auf ein Button, welches sich in der Ansicht rechts unten befindet und mit einem '+'-Symbol gekennzeichnet ist, gedrückt wird. Es erscheint ebenfalls ein Alert-Dialog, welches mithilfe eines Texteingabefeldes einen Profilnamen geben kann. Dieser Prozess kann auch abgebrochen oder bestätigt werden. Falls bei Bestätigung der Name des Profils leer oder schon vergeben ist, wird unterhalb des Textfeldes eine entsprechende Fehlermeldung ausgegeben. Wenn das Erstellen des Profils erfolgreich ist, wird das neue Profil direkt ausgewählt und bekommt die ersten acht Widgets in der Widgets View zugeordnet. Sie werden dementsprechend in der Remote Ansicht angezeigt. Alle Anpassungen, die in diesen beiden Ansichten getätigt werden, werden in den jeweilig ausgewählten Profilen gespeichert.

\section{Implementierung}
\sloppy
Der Dart-Code, welcher die Codebase für die Kompilierung des Programms darstellt, befindet sich in einem Flutter-Projekt im Verzeichnis mit dem Namen \texttt{lib}. Der Websocket wird in der \texttt{websocket\_manager.dart} verwaltet. Die Funktionalitäten der drei Views sind in den Dateien \texttt{remote\_content.dart}, \texttt{widgets\_content.dart} und \texttt{profile\_content.dart} implementiert.
\begin{figure}[h]
    \centering
    \includegraphics[width=0.2\textwidth]{pictures/flutter_directories.png}
    \captionsetup{justification=centering, labelformat=simple, singlelinecheck=false}
    \caption{Verzeichnisstruktur der \texttt{lib} im Spiegel AI Remote Projekt}
    \label{fig:flutter_directories}
\end{figure}

\subsection{Websocket}
Um die Kommunikation zwischen der Remote App und Spiegel AI sicherzustellen, muss der Websocket in der App richtig verwaltet werden. Der sogenannte \texttt{WebSocketManager} bewältigt dies mithilfe von Bibliotheken, die die Flutter-Umgebung zur Verfügung stellt. Er ermöglicht, dass eine Verbindung zum Websocket-Server hergestellt und abgebrochen werden kann und erlaubt das Empfangen und Senden von Daten. Die App empfängt Daten als Strings vom Webserver, jedoch werden nur jene mit bestimmten Eigenschaften auch verarbeitet. Der zu empfangende String muss in ein JSON-Format dekodiert werden können. Dann wird geprüft, ob der Wert des ersten Schlüssels mit der Bezeichnung \texttt{sender} den Wert \texttt{mirror} hat. Dies prüft, ob die Nachricht des Servers ursprünglich vom Spiegel AI gesendet wurde. In diesem Fall werden die Werte des Keys mit der Bezeichnung \texttt{profiles} lokal in der App gespeichert. In einem ähnlichen Stil werden Nachrichten versendet. Der einzige Unterschied ist hierbei, dass dabei der \texttt{sender} den Wert \texttt{remote} bekommt, um zu signalisieren, dass die Quelle der Nachricht die mobile Applikation ist. Gesendet werden diese immer, nachdem eine Änderung der Profile stattfindet. Es gibt jedoch eine weitere Nachricht, die die Remote App versendet. Jedes mal, nachdem eine Verbindung mit dem Websocket aufgebaut wurde, sendet sie einen String mit dem Inhalt \texttt{fetch}. Auf die Nachricht folgt, dass der Spiegel AI seinen aktuellen Stand der Profile sendet. Dies ist wichtig, damit die lokalen Profildaten der App synchronisiert werden, bevor sie in der Lage ist, Änderungen vorzunehmen. Ansonsten kann es dazu führen, dass Profildaten des Smart Mirrors mit veralteten Daten der App überschrieben werden. Um sich mit dem Websocket-Server zu verbinden, muss die IP-Adresse und der Port des Servers im Format \texttt{ws://<server-ip>:<server:port>} angegeben werden. Der Websocket schließt, sobald die App entweder geschlossen oder in den Hintergrund laufen gelassen wird. Sobald die App wieder geöffnet wird, wird auch die Verbindung zum Websocket hergestellt und sendet den \texttt{fetch}-String an den Server. \\
Ein Problem, das bei der Implementierung des Websockets besteht, ist dass die IP-Adresse des Servers im Code festgelegt ist. Das sorgt dafür, dass bei Änderung der Server-IP der Sourcecode der App umgeändert und neu kompiliert werden muss. In der Entwicklungsphase war dies noch unproblematisch, da die IP-Adresse mithilfe des Debug-Modus des Flutter Tool-Kits in kurzer Zeit zu ändern war. Beim tatsächlichen Betrieb kann der Nutzer die Remote App dadurch jedoch nicht verwenden. Ein Lösungsansatz hierfür ist, dass die App die Eingabe einer IP-Adresse zulässt. Dies kann beispielsweise im sogenannten Drawer, einer Menüansicht, implementiert werden. Hierbei kann eine "Verbinden" Option gewählt werden, welche den Nutzer dazu auffordert, eine gültige IP-Adresse einzutippen. Daraufhin versucht die Spiegel AI Remote eine Verbindung zum Websocket herzustellen. Schlägt dies fehl, wird der Nutzer erneut zur Eingabe aufgefordert. Gelingt die Verbindung zum Server, wird eine entsprechende Nachricht angezeigt. Generell ist das Implementieren von Fehlerhandling bei Verbindungsabbrüchen sinnvoll, damit der Nutzer der App sich einen besseren Überblick der Kommunikation verschaffen kann. Um herauszufinden, mit welcher IP-Adresse man sich verbinden muss, kann diese auf dem Spiegel Display angezeigt werden.

\subsection{Remote}
Die Remote App ist zuständig für die Änderung einer Profileigenschaft mit dem Namen \texttt{state}. Jedes Profil hat einen State, welcher die Positionierung (oder \texttt{index}), ID und Anzeigestatus aller ausgewählten Widgets angibt. Mithilfe dieses Status wird die Anzeige des Spiegels festgelegt. Die Änderungen des States in der Remote Ansicht sind nur am ausgewählten Profil möglich. Ist der Wert der Angegebenen ID \texttt{-1}, dann handelt es sich hierbei um ein Feld ohne Widget. Das heißt, es wird an der Stelle kein Name angezeigt und am Spiegel erscheint an dieser Position kein Widget.
\begin{figure}[h]
    \centering
    \includegraphics[width=0.35\textwidth]{pictures/remote_state.png}
    \captionsetup{justification=centering, labelformat=simple, singlelinecheck=false}
    \caption{Beispiel eines State-Eintrags im JSON-Format}
    \label{fig:remote_state}
\end{figure} \\
Die Änderung des Anzeigestatus verläuft so, dass nach dem Klick auf einen Button der Wert des Keys \texttt{enable} negiert wird. Ist der Wert \texttt{true}, so wird das visualisiert, indem die Farbe des Feldes orange ist. Ist er \texttt{false}, dass erscheint es grau. Wird ein Widget über ein anderes gezogen, so werden die Werte ihrer \texttt{index}-Schlüssel getauscht. Dies hat zur Folge, dass sowohl in der Remote Ansicht, als auch im Spiegeldisplay die Positionen geändert werden. \\
Ein Problem der Implementierung dieser Ansicht ist das mittlere Feld. Dieses war ursprünglich ein Widget, das Interaktionen zuließ. Diese Interaktionen sind nun wie gewollt nicht mehr möglich, jedoch handelt es sich hier immer noch um ein Widget, welches auch über den State versendet wird. Dies führt jedoch dazu, dass es beim Handling der Widgets viele Ausnahmesituationen gibt, in denen das mittlere Feld berücksichtigt werden muss. Ein Beispiel dafür ist das Setzen des States, wenn ein neues Profil default-Werte zugewiesen bekommt. Die IDs der Widgets, die hier ausgewählt sind, sind eine Liste der Zahlen von 0 bis 7. Um den State aber richtig zu befüllen, werden die IDs mit den Werten \texttt{0, 1, 2, 3, -1, 4, 5, 6, 7} initialisiert, da das Widget an der Stelle 4 \enquote{nicht ausgewählt} ist. Um dieses Problem zu beheben, muss das mittlere Feld aus der State genommen und das Handling der Ausnahmefälle angepasst werden.

\subsection{Widgets}
Die Widgets View verwaltet die sogenannten \texttt{selectedWidgets}. In der Liste der zehn Widgets können diese via Toggle ab- oder ausgewählt werden. Die Funktionalität der Änderung des States nachdem ein Widget selektiert oder deselektiert wurde, ist in der \texttt{\_updateProfileState(int index)} implementiert. Diese prüft zuerst, ob ein Widget ausgeschaltet wurde. In diesem Fall wird über das den State des ausgewählten Profils iteriert, bis das Widget mit der ID des \texttt{index} gefunden wurde. Die ID wird mit dem Wert \texttt{-1} überschrieben, was zur Folge hat, dass an dieser Stelle des States kein Widget mehr gewählt ist. Soll wiederum ein Widget hinzugefügt werden, wird ebenfalls durch den State iteriert, bis die ID \texttt{-1} gefunden ist. Eine Ausnahme ist hier das Widget mit an vierter Stelle, welches ignoriert werden soll. Die ID des Widgets wird daraufhin mit dem \texttt{index} des Widgets ersetzt, sodass der erste Eintrag ohne Widget nun vom selektierten Widget dargestellt wird. Hier ist auch zu beachten, dass dieser \texttt{index} nicht mit dem des States zu verwechseln ist. Es handelt sich nämlich um den \texttt{index} der Liste der Widgets, welche im State als ID dargestellt ist. \\
Ein Problem, das sich hier zeigt ist die limitierte Auswahl der Widgets. Es ist tatsächlich sehr einfach, weitere Widgets hinzuzufügen, indem man neue Einträge in der \texttt{widgetNames}-Liste anfügt. Aber die Funktionalität der Widgets muss dann in der Displayapplikation vorhanden sein.

\subsection{Profile}
Die Profile Ansicht erlaubt das Selektieren, Löschen und Hinzufügen von Profilen. Die Selektion wird durch einen Button-Klick-Listener realisiert. Das angeklickte Profil wird ausgewählt. Falls es sich jedoch um das geladene Profil handelt, wird dieses abgewählt. In dem Fall gibt es kein sogenanntes \texttt{selectedProfile} und es gibt somit keinen State, der angezeigt werden soll. Deswegen wird hier die Anzeige im Remote und im Spiegel AI auf die gleichen default-Werte gesetzt. Eine weitere Option ist das Löschen von Profilen. Nachdem die Löschung des Profils vom Nutzer bestätigt ist, wird das Profil aus der Liste der Profile entfernt und in der Profilansicht nicht mehr angezeigt. War das gelöschte Profil das \texttt{selectedProfile}, dann hat das zur Folge, dass kein Profil ausgewählt sein wird. Wird das Anlegen eines neuen Profils vom Nutzer bestätigt, so wird zunächst geprüft, ob der angegebene Name leer oder ein Duplikat ist. In den Fällen wird unterhalb des Texteingabefeldes eine jeweilige Fehlermeldung ausgegeben. Ansonsten wird das neue Profil mit dem gegebenen Namen und einer ID, welche den Millisekunden seit dem Unix-Epoch entspricht, in der Liste angelegt und selektiert. \\
In dieser View gab es öfter das Problem, dass bei Änderungen die Fehlermeldungen für den Eintrag des Namens nicht angezeigt wurden. Die genaue Ursache konnte ich nicht feststellen, jedoch schien es nicht aufzutreten, wenn der Dialog zum Hinzufügen neuer Profile unverändert blieb.

\subsection{Sonstige Implementierungen}
Eine wichtige Funktion ist das korrekte Speichern der Profile. Hierbei ist es nicht nur wichtig, dass die Profile lokal gespeichert sind, sondern dass diese auch nach jeder Speicheranweisung die Daten an den Spiegel AI sendet. Falls dies nicht geschieht, können alte Profildaten des Spiegels die Änderungen in der Remote App überschreiben. Deswegen ist es ebenfalls wichtig, dass nach jeder Profiländerung die Daten gespeichert werden. Eine weitere Funktion ist, dass die Ansichten beim Laden von Profilen aktualisiert werden. Das heißt, wenn Profile vom Websocket gesendet werden, sollen diese Änderungen in der Remote App widergespiegelt werden, ohne dass man dafür in eine andere Sicht navigieren muss. Hierbei gab es jedoch eine Schwierigkeit. Es funktionierte problemlos in der Remote und Widgets Anzeige, aber nicht im Profile View. Hier war das Problem, dass das Textfeld, welches zum Namenseintrag der Profile verwendet wurde, nicht korrekt geschlossen werden konnte, wenn eine Aktualisierung der Ansicht stattfand. Dieser Fehler war besonders frustrierend, da ich den Textfeld-Controller tatsächlich mit dem \texttt{dispose()}-Befehl schließe. Trotz Debugging und Recherche kam ich zu keiner Lösung dieses Problems und entschied mich, die Profilanzeige nicht zu aktualisieren, wenn der Spiegel Profile sendet.
