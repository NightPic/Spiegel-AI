\chapter{Ergebnisse}

In diesem Kapitel fassen wir die Ergebnisse des Projekts \textbf{Spiegel AI} zusammen. Wir gehen auf die erreichten Ziele, die Herausforderungen und die zukünftigen Arbeiten ein.

\section{Zusammenfassung der Ergebnisse}

\subsection{Erreichte Ziele}
Unser Projektteam hat den Smart Mirror erfolgreich realisiert und dabei folgende Ziele erreicht:

Der Spiegelprototyp wurde ansehnlich und robust aufgebaut. Er ist nicht nur funktional, sondern fügt sich auch optisch ansprechend in die Umgebung ein. Die Display-Ausgabe funktioniert einwandfrei und ermöglicht eine klare und deutliche Darstellung aller Inhalte.

Der Smart Mirror ist in der Lage, die aktuelle Wetterlage des Ortes sowie Kalendereinträge anzuzeigen und bietet viele weitere informative und nützliche Funktionen. Benutzer des Spiegels haben die Möglichkeit, ihr angezeigtes Profil individuell zu gestalten. Dies erfolgt durch die Auswahl aus neun verschiedenen Widgets, die auf acht unterschiedlichen Positionen platziert werden können. Diese Personalisierung ermöglicht eine hohe Flexibilität und Anpassungsfähigkeit an die individuellen Bedürfnisse und Vorlieben der Benutzer.

Ein weiteres herausragendes Merkmal ist die interaktive Steuerung über eine mobile Kontroll-App. Diese App ermöglicht es den Benutzern, ihr Profil zu ändern oder zu personalisieren, was die Benutzerfreundlichkeit und den Komfort erheblich steigert. Alle Informationen werden in deutscher Sprache angezeigt, was die Bedienung für deutschsprachige Nutzer intuitiv und einfach macht.

Ein neues Gesicht wird erkannt und das jeweilige Profil angezeigt, was eine schnelle und reibungslose Nutzung sicherstellt. Die Benutzeroberfläche wurde in Tests als 'sehr intuitiv' eingestuft, was die hohe Benutzerfreundlichkeit des Systems unterstreicht.

Technisch konnten wir sicherstellen, dass sowohl der Spiegel als auch die App erfolgreich eine Verbindung zum Websocket herstellen können. Beide Komponenten senden nach einer ausgeklügelten Synchronisationslogik jeweils den aktuellen Stand der Profile, wodurch eine konsistente und aktuelle Datenbasis gewährleistet wird. Die Gesichtserkennung arbeitet äußerst zuverlässig und erkennt Personen zu 95 Prozent korrekt, wodurch die personalisierten Profile geladen und angezeigt werden können.

\subsection{Herausforderungen}
Die Herausforderungen, denen sich das Team während der Entwicklung des Projekts stellen musste, sind in den jeweiligen Dokumentationen der einzelnen Projektteilnehmer ausführlich beschrieben. 

\subsection{Zukünftige Arbeiten}
Für zukünftige Arbeiten und mögliche Erweiterungen des Projekts verweisen wir ebenfalls auf die detaillierten Ausführungen in den Dokumentationen der einzelnen Projektteilnehmer.

\subsection{Betriebssetzung}
Zum Verwenden des Spiegel AIs sind einige Schritte notwendig. Zuerst benötigen Sie eine Kamera und einen WLAN-Stick, da diese Komponenten Eigentum der Projektteilnehmer waren, welche sie wieder mitnahmen. Beim Verwenden eines WLAN-Sticks ist noch zu beachten, dass gegebenenfalls Treiber installiert werden müssen. Nachdem die Hardware bereit ist, sind die folgenden Schritte zur Betriebssetzung des Smart Mirrors nötig:
\begin{enumerate}
	\item Schließen Sie alle über USB zu verbindenden Komponenten, inklusive Tastatur und Maus, am Raspberry Pi an.
	\item Schalten Sie den Raspberry Pi ein
	\item Öffnen Sie ein Terminal-Fenster und navigieren Sie zu \verb|~/Mirror/dt-g5/|
	\item Öffnen Sie zwei neue Tabs
	\item Navigieren Sie im ersten Tab zu \verb|~/Mirror/dt-g5/websocket_server| und starten Sie den Websocket mit \texttt{python websocket.py}
	\item Navigieren Sie im zweiten Tab zu \verb|~/Mirror/dt-g5/Display| und starten Sie einen HTTP-Server mit \texttt{python -m http.server 8001}
	\item Navigieren Sie im dritten Tab zu \\ \verb|~/Mirror/dt-g5/facialRec/facialRecReadability/mitWebsocket| und starten Sie die Gesichtserkennung mit \texttt{python main.py}
	\item Öffnen Sie den Chromium Browser und rufen Sie \texttt{http://localhost:8001} auf
	\item Eventuell müssen Sie hier den Cache bereinigen, falls die Anzeige sich nicht ändert
	\item Installieren Sie die Android-APK auf einem Smartphone. Sie ist im GitLab-Repository im Verzeichnis \texttt{remote\_app} hinterlegt
	\item Starten Sie die Applikation auf dem Smartphone
\end{enumerate}
